\section{Profile}

\begin{multicols}{2}

Die Profile in \sscontrol{} beschreiben die Umgebungen in denen \sscontrol{} die
Services einrichten sol.

Ein typisches Profil enthält einen Namen, wie ``ubuntu'', und eine Auflistung
der Services, die \sscontrol{} einrichten und verwalten soll. Alle diese
Services wird \sscontrol{} verwalten, alle anderen Services sind für
\sscontrol{} unbekannt.

\end{multicols}

\subsection{Profile Direktiven}

\begin{multicols}{2}

Ein Profilscript wird eingeleitet durch die \directive{profiles}.
Danach wird ein neues Profil angelegt durch die
\directive{create\_profile("name")}. Der \parameter{name} bezeichnet das Profil
und kann durch diesen Namen ausgewählt werden. Danach kommt eine Auflistung der
Services, die \sscontrol{} verwalten soll.

\paragraph{Beispiel Profil Ubuntu Server}

Wie man durch das Beispiel sieht, anthält das Profil die Konfiguration der
Services, wo sich bestimmte Kommandozeilentools befinden und wo die
Konfiguration für die Services gespeichert wird. Die Services enthalten meistens
die Direktiven:
\begin{inparaitem}
\item \directive{configuration\_directory},
\item \directive{restart\_command},
\item \directive{status\_command}
\end{inparaitem}. 

Außerdem können die Direktiven auch Variablen enthalten, hervorgedrückt durch
das Paarweise ``\$'' Symbol. Diese Variablen werden später durch die Services
selbst ersetzt.

Die Bedeutung der einzelnen Servicedirektiven werden später in den Abschnitten
der Services erklärt.

\end{multicols}

\begin{lstlisting}[style=Java, caption=Beispiel Profil Ubuntu Server]
profiles {
    create_profile("ubuntu") {
        hostname {
            configuration_directory '/etc'
            restart_command '/etc/init.d/hostname restart'
            status_command '/bin/hostname'
        }
        hosts {
            configuration_directory '/etc'
            service ''
            status_command '/bin/cat /etc/hosts'
        }
        network {
            configuration_directory '/etc/network'
            restart_command '/etc/init.d/networking restart'
            status_command '/bin/cat /etc/network/interfaces'
        }
        httpd {
            configuration_directory '/etc/apache2'
            restart_command '/etc/init.d/apache2 restart'
            status_command '/usr/sbin/apache2ctl configtest'
            domain_web_directory_prefix '/var/www/sites/$domain_name$/web'
            domain_ssl_directory_prefix '/var/www/sites/$domain_name$/ssl'
            domain_page_directory_prefix '/var/www/sites/$domain_name$/$page_name$/web'
        }
        database {
            configuration_directory '/etc/mysql/conf.d'
            restart_command '/sbin/restart mysql'
            status_command '/sbin/status mysql'
        }
    }
}
\end{lstlisting}

