\subsection{Mail Script}

\paragraph{mail}

\statement{mail}{mail \{...\}}

Entry point in the mail script.

\paragraph{name}

\statement{mail!name}{name \typestring}

The domain name of the server.

\paragraph{origin}

\statement{mail!origin}{origin \typestring}

The domain name that locally-posted mail appears to come
from, and that locally posted mail is delivered to.

\paragraph{bind\_addresses}

\property{mail!bind\_addresses}{bind\_addresses \literalall$|$\literalloopback$|$\typestring}

The network interface addresses that this mail system receives mail on.
Specify \literalall{} to receive mail on all network interfaces,
and \literalloopback{} to receive mail on loopback network interfaces only.
Defaults to \literalall.

\paragraph{masquerade}

\statement*{masquerade \{ }\\
\statement*{[domain \typelist] }\\
\statement*{[user \typelist] \} }

\paragraph{masquerade}

\statement{mail!masquerade}{masquerade \{ \dots \}}

Starts a list of domains and users to masquerade. Masqueraded domains and
users are stripped off of their subdomain structure in email addresses.
A domain name prefixed with ``!'' means do not masquerade this domain
or its subdomains.

\paragraph{domain}

\statement{mail!masquerade!domain}{domain \typelist}

Sets the domains to masquerade.

\paragraph{user}

\statement{mail!masquerade!user}{user \typelist}

Sets the list of user names that are not subjected to address masquerading.

\paragraph{certificate}

\statement*{certificate }\\
\statement*{file: \typestring,}\\
\statement*{key: \typestring,}\\
\statement*{ca: \typestring}

Defines the location of the certificate files for TLS.

\paragraph{file}

\statement{mail!certificate!file}{file \typestring{}}

The location of the certificate file.

\paragraph{key}

\statement{mail!certificate!key}{key \typestring{}}

The location of the certificate key file.

\paragraph{ca}

\statement{mail!certificate!ca}{ca \typestring{}}

The location of the certificate CA file.

\paragraph{domain}

\statement*{domain \typestring{} \{}\\
\statement*{catchall \dots}\\
\statement*{alias \dots}\\
\statement*{user \dots}\\
\statement*{enabled \typeboolean}\\
\statement*{\}}

Adds domains that are known to the mail server. A domain can have a catch-all
alias, other aliases, users and an enabled flag. The current domain is appended
to aliases and users without a domain part. So for example the user ``xandros''
in the ``blobber.org'' domain will have the address ``xandros@blobber.org''.

\paragraph{catchall}

\statement{mail!domain!catchall}{catchall destination: \typestring{}}

Creates a catch-all alias to the current destination. The catch-all is equal
to the alias ``@domain'', where the ``domain'' is the current domain name.

Example:

\begin{compactitem}
\item ``\code{catchall destination: "@blobber.org"}''
\item ``\code{catchall destination: "xandros@blobber.org"}''
\end{compactitem}

\paragraph{alias}

\statement*{alias \typestring{}, destination \typestring}\\
\statement*{[\{ enabled \typeboolean \}]}\\

\paragraph{alias}

\statement{mail!domain!alias}{alias \typestring, destination: \typestring{}}

Creates a new alias with the specified destination. The current domain is appended
to the aliases without a domain part. So for example the alias ``karl''
in the ``blobber.org'' domain will have the address ``karl@blobber.org''.

Example:

\begin{compactitem}
\item ``\code{alias "karl", destination: "karl.vovianda@galias.com"}''
\item ``\code{alias "postmaster@blobber.org", destination: "postmaster@\\localhost"}''
\end{compactitem}

\paragraph{enabled}

\statement{mail!domain!enabled}{enabled \typeboolean}

Sets the enabled flag on a domain, user or alias. Disabled domains, users or
aliases are ignored by the mail server.

\paragraph{user}

\statement*{user \typestring{}, password \typestring}\\
\statement*{[\{ enabled \typeboolean \}]}\\

\paragraph{user}

\statement{mail!domain!user}{user \typestring, password: \typestring{}}

Adds a new user with the specified passwrod for the domain.
The current domain is appended
to the users without a domain part. So for example the user ``karl''
in the ``blobber.org'' domain will have the address ``karl@blobber.org''.

