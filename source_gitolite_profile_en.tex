\label{sec:gitolite_profile}
\subsection{Gitolite Profile}

The Gitolite\footnote{\url{http://gitolite.com/}} hosting service can fully
manage a Git\footnote{\url{http://git-scm.com/}} repository on the server.
It allowes for fine grained access control per repository.

For the distribution \TheDistribution{ubuntu} Ubuntu 12.04 the Gitolite
service is downloaded and installed into the prefix directory. The local
user and group is created or updated and is used under which the the service
is run. The repositories directory is created under the data directory path
and the owner is set to the service user and group name, and the file permissions
are updated based on the specified umask. The structure of the service on the
file system will be as follow:

\begin{compactitem}
%
\item \code{/usr/local/gitolite}
Gitolite service files, owner is root, as set 
in the \Statement*[source:setup:gitolite:prefix]{prefix} statement and in 
the \Property*[source:gitolite:gitolite_default_prefix]{gitolite\_default\_prefix} property;
%
\item \code{/var/git/.gitolite.rc}
Gitolite configuration file, owner is Gitolite user and group, as set 
in \Statement*[source:setup:gitolite:user]{user} statement and in 
the \Property*[source:gitolite:user_add_command]{user\_add\_command} property;
%
\item \code{/var/git/repositories}
repositories parent directory, owner is Gitolite user and group, permissions set 
to ([0777]-umask), set 
in \Property*[source:gitolite:gitolite_umask]{gitolite\_umask} property.
\end{compactitem}

After the installation, the Gitolite service is available and the administration
repository can be downloaded via the 
command \qcode{git clone git@server.com:gitolite-admin}. The user name is set
in the \Statement*[source:setup:gitolite:user]{user} statement and in 
the \Property*[source:gitolite:gitolite_default_user]{gitolite\_\\default\_user} property.

\subsubsection{Commands}

%% install_command
\TheStatement[source:gitolite:install_command]{install\_\\command}
\TheStatement*[source!gitolite!install\_command]{install\_command \Arg{command}}

The \Arg{command} to install packages on the system. Can be a full path or
just the command name that can be found in the current search path. 
For the distribution
\begin{inparaitem}
\item[\TheDistribution{ubuntu}] Ubuntu 12.04
\end{inparaitem}
the value is set to \qcode{/usr/bin/aptitude}.

%% su_command
\TheStatement[source:gitolite:su_command]{su\_command}
\TheStatement*[source!gitolite!su\_command]{su\_command \Arg{command}}

The \Arg{command} to change local user under which a scrip is run. Can be a full path or
just the command name that can be found in the current search path. 
For the distribution
\begin{inparaitem}
\item[\TheDistribution{ubuntu}] Ubuntu 12.04
\end{inparaitem}
the value is set to \qcode{/bin/su}.

%% chmod_command
\TheStatement[source:gitolite:chmod_command]{chmod\_command}
\TheStatement*[source!gitolite!chmod\_command]{chmod\_command \Arg{command}}

The \Arg{command} to the change permissions command. Can be a full path or
just the command name that can be found in the current search path. 
For the distribution
\begin{inparaitem}
\item[\TheDistribution{ubuntu}] Ubuntu 12.04
\end{inparaitem}
the value is set to \qcode{/bin/chmod}.

%% chown_command
\TheStatement[source:gitolite:chown_command]{chown\_command}
\TheStatement*[source!gitolite!chown\_command]{chown\_command \Arg{command}}

The \Arg{command} to the change permissions command. Can be a full path or
just the command name that can be found in the current search path. 
For the distribution
\begin{inparaitem}
\item[\TheDistribution{ubuntu}] Ubuntu 12.04
\end{inparaitem}
the value is set to \qcode{/bin/chown}.

%% group_add_command
\TheStatement[source:gitolite:group_add_command]{group\_\\add\_command}
\TheStatement*[source!gitolite!group\_add\_command]{group\_add\_command \Arg{command}}

The \Arg{command} to create a new local group. Can be a full path or
just the command name that can be found in the current search path. 
For the distribution
\begin{inparaitem}
\item[\TheDistribution{ubuntu}] Ubuntu 12.04
\end{inparaitem}
the value is set to \qcode{/usr/sbin/groupadd}.

%% group_mod_command
\TheStatement[source:gitolite:group_mod_command]{group\_\\mod\_command}
\TheStatement*[source!gitolite!group\_mod\_command]{group\_mod\_command \Arg{command}}

The \Arg{command} to modify a local group. Can be a full path or
just the command name that can be found in the current search path. 
For the distribution
\begin{inparaitem}
\item[\TheDistribution{ubuntu}] Ubuntu 12.04
\end{inparaitem}
the value is set to \qcode{/usr/sbin/groupmod}.

%% user_add_command
\TheStatement[source:gitolite:user_add_command]{user\_\\add\_command}
\TheStatement*[source!gitolite!user\_add\_command]{user\_add\_command \Arg{command}}

Returns the \Arg{command} to create a new local user. Can be a full path or
just the command name that can be found in the current search path. 
For the distribution
\begin{inparaitem}
\item[\TheDistribution{ubuntu}] Ubuntu 12.04
\end{inparaitem}
the value is set to \qcode{/usr/sbin/useradd}.

%% user_mod_command
\TheStatement[source:gitolite:user_mod_command]{user\_\\mod\_command}
\TheStatement*[source!gitolite!user\_mod\_command]{user\_mod\_command \Arg{command}}

The \Arg{command} to modify a local user. Can be a full path or
just the command name that can be found in the current search path. 
For the distribution
\begin{inparaitem}
\item[\TheDistribution{ubuntu}] Ubuntu 12.04
\end{inparaitem}
the value is set to \qcode{/usr/sbin/usermod}.

%% tar_command
\TheStatement[source:gitolite:tar_command]{tar\_command}
\TheStatement*[source!gitolite!tar\_command]{tar\_command \Arg{command}}

The \Arg{command} of the \code{tar} utility. Can be a full path or
just the command name that can be found in the current search path. 
For the distribution
\begin{inparaitem}
\item[\TheDistribution{ubuntu}] Ubuntu 12.04
\end{inparaitem}
the value is set to \qcode{/bin/tar}.

%% gitolite_install_command
\TheStatement[source:gitolite:gitolite_install_command]{gitolite\_\\install\_command}
\TheStatement*[source!gitolite!gitolite\_install\_command]{gitolite\_install\_command \Arg{path}}

The \Arg{path} of the Gitolite install command. If the path is not absolute it
is assumed to be located under the service installation directory.
For the distribution
\begin{inparaitem}
\item[\TheDistribution{ubuntu}] Ubuntu 12.04
\end{inparaitem}
the value is set to \qcode{install}.

%% gitolite_command
\TheStatement[source:gitolite:gitolite_command]{gitolite\_\\command}
\TheStatement*[source!gitolite!gitolite\_command]{gitolite\_command \Arg{path}}

The \Arg{path} of the Gitolite command. If the path is not absolute it
is assumed to be located under the service installation directory.
For the distribution
\begin{inparaitem}
\item[\TheDistribution{ubuntu}] Ubuntu 12.04
\end{inparaitem}
the value is set to \qcode{../bin/gitolite}.

\subsubsection{Files and Directories}

%% temp_directory
\TheStatement[source:gitolite:temp_directory]{temp\_\\directory}
\TheStatement*[source!gitolite!temp\_directory]{temp\_directory \Arg{path}}

Returns the \Arg{path} of the temporary directory, if empty, the path
of the system temporary directory is used.
For the distribution
\begin{inparaitem}
\item[\TheDistribution{ubuntu}] Ubuntu 12.04
\end{inparaitem}
the value is set to \qcode{}, empty.

%% groups_file
\TheStatement[source:gitolite:groups_file]{groups\_file}
\TheStatement*[source!gitolite!groups\_file]{groups\_file \Arg{path}}

Returns the \Arg{path} of the local groups file.
For the distribution
\begin{inparaitem}
\item[\TheDistribution{ubuntu}] Ubuntu 12.04
\end{inparaitem}
the value is set to \qcode{/etc/group}.

%% users_file
\TheStatement[source:gitolite:users_file]{users\_file}
\TheStatement*[source!gitolite!users\_file]{users\_file \Arg{path}}

Returns the \Arg{path} of the local users file.
For the distribution
\begin{inparaitem}
\item[\TheDistribution{ubuntu}] Ubuntu 12.04
\end{inparaitem}
the value is set to \qcode{/etc/passwd}.

%% gitolite_repositories_directory
\TheStatement[source:gitolite:gitolite_repositories_directory]{gitolite\_\\repositories\_\\directory}
\TheStatement*[source!gitolite!gitolite\_repositories\_directory]{gitolite\_repositories\_directory \Arg{path}}

Returns the \Arg{path} of the repositories directory. If the path is not
absolute it is assumed to be located under the service
data directory.
For the distribution
\begin{inparaitem}
\item[\TheDistribution{ubuntu}] Ubuntu 12.04
\end{inparaitem}
the value is set to \qcode{repositories}.

%% gitolite_rc_file
\TheStatement[source:gitolite:gitolite_rc_file]{gitolite\_\\rc\_file}
\TheStatement*[source!gitolite!gitolite\_rc\_file]{gitolite\_rc\_file \Arg{path}}

Returns the \Arg{path} of the \qcode{.gitolite.rc} file. If the path is not
absolute it is assumed to be located under the service
data directory.
For the distribution
\begin{inparaitem}
\item[\TheDistribution{ubuntu}] Ubuntu 12.04
\end{inparaitem}
the value is set to \qcode{.gitolite.rc}.

\subsubsection{Other Properties}

%% system_name
\TheStatement[source:gitolite:system_name]{system\_name}
\TheStatement*[source!gitolite!system\_name]{system\_name \Arg{name}}

The system \Arg{name}. 
For the distribution
\begin{inparaitem}
\item[\TheDistribution{ubuntu}] Ubuntu 12.04
\end{inparaitem}
the value is set to \qcode{ubuntu}.

%% charset
\TheStatement[source:gitolite:charset]{charset}
\TheStatement*[source!gitolite!charset]{charset \Arg{name}}

The character set \Arg{name} of the service configuration files. 
For the distribution
\begin{inparaitem}
\item[\TheDistribution{ubuntu}] Ubuntu 12.04
\end{inparaitem}
the value is set to \qcode{utf-8}.

%% gitolite_packages
\TheStatement[source:gitolite:gitolite_packages]{gitolite\_\\packages}
\TheStatement*[source!gitolite!gitolite\_packages]{gitolite\_packages \Arg{packages}}

The list of \Arg{packages} that are needed for the service.
For the distribution
\begin{inparaitem}
\item[\TheDistribution{ubuntu}] Ubuntu 12.04
\end{inparaitem}
the value is set to \qcode{git, openssh-server, perl}.

%% gitolite_umask
\TheStatement[source:gitolite:gitolite_umask]{gitolite\_\\umask}
\TheStatement*[source!gitolite!gitolite\_umask]{gitolite\_umask \Arg{mask}}

The file creation \Arg{mask} for new repositories.
For the distribution
\begin{inparaitem}
\item[\TheDistribution{ubuntu}] Ubuntu 12.04
\end{inparaitem}
the value is set to \qcode{0077}.

%% gitolite_default_prefix
\TheStatement[source:gitolite:gitolite_default_prefix]{gitolite\_\\default\_\\prefix}
\TheStatement*[source!gitolite!gitolite\_default\_prefix]{gitolite\_default\_prefix \Arg{path}}

The default service prefix \Arg{path}. That is the path where the
service is installed to. 
For the distribution
\begin{inparaitem}
\item[\TheDistribution{ubuntu}] Ubuntu 12.04
\end{inparaitem}
the value is set to \qcode{/usr/local/gitolite}.

%% gitolite_default_data
\TheStatement[source:gitolite:gitolite_default_data]{gitolite\_\\default\_\\data}
\TheStatement*[source!gitolite!gitolite\_default\_data]{gitolite\_default\_data \Arg{path}}

The default service prefix \Arg{path}. That is the path where the
service is installed to. 
For the distribution
\begin{inparaitem}
\item[\TheDistribution{ubuntu}] Ubuntu 12.04
\end{inparaitem}
the value is set to \qcode{/var/git}.

% gitolite_default_override_mode
\TheStatement[source:gitolite:gitolite_default_override_mode]{gitolite\_\\default\_\\override\_mode}
\TheStatement*[source!gitolite!gitolite\_default\_override\_mode]{gitolite\_default\_override\_mode \Arg{mode}}

The default override \Arg{mode}.
For the distribution
\begin{inparaitem}
\item[\TheDistribution{ubuntu}] Ubuntu 12.04
\end{inparaitem}
the value is set to \qcode{upgrade}.

% gitolite_default_user
\TheStatement[source:gitolite:gitolite_default_user]{gitolite\_\\default\_user}
\TheStatement*[source!gitolite!gitolite\_default\_user]{gitolite\_default\_user \Arg{name}}

The default service local user \Arg{name}.
For the distribution
\begin{inparaitem}
\item[\TheDistribution{ubuntu}] Ubuntu 12.04
\end{inparaitem}
the value is set to \qcode{git}.

% gitolite_default_group
\TheStatement[source:gitolite:gitolite_default_group]{gitolite\_\\default\_group}
\TheStatement*[source!gitolite!gitolite\_default\_group]{gitolite\_default\_group \Arg{name}}

The default service local group \Arg{name}.
For the distribution
\begin{inparaitem}
\item[\TheDistribution{ubuntu}] Ubuntu 12.04
\end{inparaitem}
the value is set to \qcode{git}.
