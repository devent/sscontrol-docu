\section{Command Line Interface}

In den ersten Versionen wird das Programm durch ein Command Line Interface
benutzt. Die Kommandozeilenprogram nimmt Skripte entgegen und Konfiguriert den
Zielserver damit.

\code{sscontrol -profile <profile-script> -script <script-name> -server
<server-ip>}

Der \code{-profile <profile-script>} Parameter wird ein Serverprofil laden.

\begin{asparaitem}
\item \code{ubuntu\_profile.groovy}
\item \code{fedora\_profile.groovy}
\end{asparaitem}

Der \code{-script <script-name>} Parameter setzt das Verzeichnis in dem die
Server-Scripte aufbewart sind. Dort ist auch das Serverprofil zu finden.

Der \code{-server <server-ip>} Parameter setzt den Server der Konfiguriert
werden soll. Der Server wird anhand der IP-Adresse identifiziert.
