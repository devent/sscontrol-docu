\section{Ziele}

Das Ziel von \sscontrol{} ist es durch ein Script zu sagen welche Serverservices
eingerichtet werden sollen und \sscontrol{} soll vollständig in der Lage sein
diese Services einzurichten.

Als Beispiel, wenn man dem \sscontrol{} durch ein Script eingibt einen Webserver
einzurichten, dass mehrere Domains mit jeweils Drupal Seiten verwalten
soll, dann enthält das Script nicht wie man einen Web Server einrichtet,
sondern dass man einen Web Server haben will. \sscontrol{} wird sich
eigenständig darum kümmern die Pakate zu installieren und die Domains mit den
Drupal Seiten einzurichten.

\sscontrol{} verwendet Konvention über Konfiguration mit der Hilfestellung von
Profilen. Die Profile werden für den Zielserver eingerichtet und enthalten
Informationen über die Umgebung und über die einzusetzenden Services.

Zum Beispiel, ein Ubuntu Server muss anderes eingerichtet sein als ein RedHat
Server, und Apache2 muss anderes eingerichtet werden als Lighthttpd.

Diese Profile muss man nur einmal erstellen und dann kann man sie für
verschiedene Konfigurationen des \sscontrol{} verwenden. Als Beispiel kann man
das gleiche Profil verwenden um einen Webserver mit Domains in denen Drupal
Seiten eingerichtet werden sollen und einen Dateiserver einzurichten, der FTP
Zugriff für Benutzer liefert.

