\subsubsection{Cgit Profil}

\begin{lstlisting}[style=Java, caption=Beispiel Cgit Profil]
cgitGitUrl = "git://hjemli.net/pub/git/cgit"
cgitGitBranch = "stable"

server {
  profiles {
    create_profile("ubuntu") {
      system {
        packages_install_command "export DEBIAN_FRONTEND=noninteractive; /usr/bin/apt-get -q -y update && /usr/bin/apt-get -q -y install"
        tmp_directory "/tmp"
        chown_command "/bin/chown -R"
        chmod_command "/bin/chmod -R"
        cd_command "cd"
        usermod_command "/usr/sbin/usermod"
        git_command "/usr/bin/git"
        make_command "/usr/bin/make"
      }
      httpd {
        handler "apache2"
        configuration_directory "/etc/apache2"
        restart_command "/etc/init.d/apache2 restart"
        status_command "/usr/sbin/apache2ctl configtest"
        status_command_expected ".*Syntax OK.*"
        packages "apache2"

        domain_web_directory_prefix "/var/www/%domain_name%/web"
        domain_web_group "www-data"
        domain_web_owner "www-data"
        domain_web_directory_mod "a-rx,o-w"
        domain_ssl_directory_prefix "/var/www/%domain_name%/ssl"
        domain_ssl_group "root"
        domain_ssl_owner "root"
        domain_ssl_directory_mod "o-wrx"

        apache2_a2enmod_command "/usr/sbin/a2enmod"
        apache2_mods "ssl, rewrite"

        cgit_packages "build-essential, git-core, libssl-dev, libcurl4-openssl-dev"
        cgit_user "www-data"
        cgit_group "www-data"
        cgit_git_url cgitGitUrl
        cgit_git_branch cgitGitBranch
        cgit_cache_root_prefix "/var/cache/cgit"
        cgit_update_user true
      }
    }
  }
}
\end{lstlisting}

Die Cgit Profil-Directiven befinden sich im httpd-Block und haben das
cgit-Prefix. Das Beispiel zeigt eine Konfiguration für einen Ubuntu 10.10 Server
mit Apache2 als den httpd-Server.

Es werden die folgenden System-Direktiven erwartet:

\paragraph{packages\_install\_command}

\directive{packages\_install\_command} erwartet eine Zeichenfolge, die das Kommando
bezeichnen, dass verwendet wird um neue Pakete zu installieren.

\paragraph{tmp\_directory}

\directive{tmp\_directory} erwartet eine Zeichenfolge, die das Verzeichnis bezeichnet
das für temporäre Dateien benutzt werden kann.

\paragraph{chmod\_command}

\directive{chmod\_command} erwartet eine Zeichenfolge, die das
chmod-Kommando bezeichnen. Es ist nötig, wenn man Cgit von dem Git-Repository installieren will.

\paragraph{chown\_command}

\directive{chown\_command} erwartet eine Zeichenfolge, die das
chown-Kommando bezeichnen.

Damit wird der Benutzername und Gruppe des Cache-Root-Verzeichnis zugewiesen.

\paragraph{cd\_command}

\directive{cd\_command} erwartet eine Zeichenfolge, die das
cd-Kommando bezeichnen. Es ist nötig, wenn man Cgit von dem Git-Repository installieren will.

\paragraph{usermod\_command}

\directive{usermod\_command} erwartet eine Zeichenfolge, die das
usermod-Kommando bezeichnen. Es muss dafür gesorgen werden, dass der Apache2
Server auch die Git-Repositories lesen kann damit Cgit sie anzeigen kann. Das
Programm wird automatisch den Apache2-Benutzer zu den Gruppen hinzufügen, die
Lesezugriff auf die Git-Repositories haben, wenn die
\directive{cgit\_update\_apache\_user} auf \code{true} gesetzt ist.

\paragraph{git\_command}

\directive{git\_command} erwartet eine Zeichenfolge, die das git-Kommando bezeichnet.
Es ist nötig, wenn man Cgit von dem Git-Repository installieren will.

\paragraph{make\_command}

\directive{make\_command} erwartet eine Zeichenfolge, die das make-Kommando bezeichnet.
Es ist nötig, wenn man Cgit von den Source-Quellen installieren will.

Es werden weitere Direktiven benötigt:

\paragraph{cgit\_user}

\directive{cgit\_user} erwartet eine Zeichenfolge, die den Benutzernamen
bezeichnet, mit dem das Cgit Script läuft. Dieser Benutzer muss Lesezugriff auf
die Repositories haben.

Diesem Benutzer wird auch das Cache-Root-Verzeichnis zugewiesen.

\paragraph{cgit\_group}

\directive{cgit\_group} erwartet eine Zeichenfolge, die den Gruppennamen
bezeichnet, mit dem das Cgit Script läuft. Diese Gruppe muss Lesezugriff auf
die Repositories haben.

Dieser Gruppe wird auch das Cache-Root-Verzeichnis zugewiesen.

\paragraph{cgit\_git\_url}

Die optionale \directive{cgit\_git\_url} erwartet eine Zeichenfolge, die die URL
des Git-Repository angibt. Diese Direktive ist nur nötig, wenn
das System kein Cgit-Paket anbietet und man von dem Git-Repository installieren will.

\paragraph{cgit\_git\_branch}

Die optionale \directive{cgit\_git\_branch} erwartet eine Zeichenfolge, die den Zweig-Namen
im Git-Repository angibt. Diese Direktive ist nur nötig, wenn
das System kein Cgit-Paket anbietet und man von dem Git-Repository installieren will.

Das Cgit-Repository ist auf \url{http://hjemli.net/git/cgit/} zu finden, dort kann man
die aktuellen Zweige sehen.

\paragraph{cgit\_url}

Die optionale \directive{cgit\_url} erwartet eine Zeichenfolge, die die URL
enthält von wo man Cgit herunter laden kann. Diese Direktive ist nur nötig, wenn
das System kein Cgit-Paket anbietet.

\paragraph{cgit\_hash\_url}

Die optionale \directive{cgit\_hash\_url} erwartet eine Zeichenfolge, die die
URL enthält von wo man das Hash zum Cgit-Archiv herunter laden kann. Diese
Direktive ist nur nötig, wenn das System kein Cgit-Paket anbietet und man das
Cgit-Archiv automatisch überprüfen will. Auf der Homepage von Cgit befindet sich
leider kein Hash-Datei zum herunter laden, aber man kann einmal eine Hash-Datei
erstellen und diese benutzen.

\paragraph{cgit\_packages}

Die \directive{cgit\_packages} erwartet eine Liste von Paketen, die man benötigt
um Cgit zu installieren. Falls Cgit von dem System als ein Paket angeboten wird,
dann kann hier der Paketname eingetragen werden.

Ubuntu 10.10 enthält kein Cgit-Paket, somit muss Cgit aus dem Quellcode
kompiliert werden. Dazu werden die folgenden Pakete benötigt:
\begin{asparaitem}
\item \code{build-essential},
\item \code{git-core},
\item \code{libssl-dev},
\item \code{libcurl4-openssl-dev}.
\end{asparaitem}

\paragraph{cgit\_cache\_root\_prefix}

Die \directive{cgit\_cache\_root\_prefix} erwartet eine Zeichenfolge, die das
Verzeichnis ist, wo Cgit sein Cache speichern kann. Für jede Cgit Installation
wird ein neues Unterverzeichnis unter diesem Verzeichnis erstellt, damit die
Installationen sich nicht gegenseitig beeinflussen können.

Wenn z.B. das Prefix \code{/var/cache/cgit} ist und es zwei Cgit Installation, gitpublic und gitprivate,
gibt, dann werden \code{/var/cache/cgit/gitpublic} und \code{/var/cache/cgit/gitprivate} als Cache-Root-Verzeichnisse
benutzt.

\paragraph{cgit_update_user}

Die \directive{cgit_update_user} erwartet entweder \code{true} oder
\code{false}, je nach dem ob das Programm automatisch dem Cgit-Benutzer zu
den Gruppen hinzufügen soll, die Lesezugriff auf die Git-Repositories haben.

Falls der Wert \code{true} ist, dann wird der in der
\directive{usermod\_command} Kommando benutzt, um den Benutzer zu den
Repositories-Gruppen hinzuzufügen.
