\subsection{Hostname Profile}

The hostname comes in the distributions 
\begin{compactitem}
\item[\TheDistribution{ubuntu}] Ubuntu 10.04\footnote{\TheUbuntuMaverickLTSDate} in version 3.03 in the main repository;
\item[\TheDistribution{ubuntu}] Ubuntu 12.04\footnote{\TheUbuntuPreciseLTSDate} in version 3.06 in the main repository;
\end{compactitem}

%% install_command
\TheProperty[hostname:install_command]{install\_\\command}
\TheProperty*[hostname!install\_command]{install\_command \Arg{command}}

The \Arg{command} to install package on the system. Can be a full path or
just the command name that can be found in the current search path. 
For the distributions 
\begin{inparaitem}
\item[\TheDistribution{ubuntu}] Ubuntu 10.04,
\item[\TheDistribution{ubuntu}] Ubuntu 12.04
\end{inparaitem}
the value is set to \qcode{export DEBIAN\_FRONTEND=noninteractive\textbackslash{n} /usr/bin/aptitude -q update \&\& /usr/bin/aptitude -q -y install}.

%% restart_command
\TheProperty[hostname:restart_command]{restart\_\\command}
\TheProperty*[hostname!restart\_command]{restart\_command \Arg{command}}

The \Arg{command} to reload the hostname. Can be a full path or
just the command name that can be found in the current search path.
For the distribution
\begin{inparaitem}
\item[\TheDistribution{ubuntu}] Ubuntu 10.04,
\item[\TheDistribution{ubuntu}] Ubuntu 12.04
\end{inparaitem}
the value is set to \qcode{/etc/init.d/hostname restart}.

%% configuration_directory
\TheProperty[hostname:configuration_directory]{configuration\_\\directory}
\TheProperty*[hostname!configuration\_directory]{configuration\_directory \Arg{command}}

The path of the hostname configuration directory. Can be a full path or
just the command name that can be found in the current search path.
For the distribution
\begin{inparaitem}
\item[\TheDistribution{ubuntu}] Ubuntu 10.04,
\item[\TheDistribution{ubuntu}] Ubuntu 12.04
\end{inparaitem}
the value is set to \qcode{/etc}, so that the hostname configuration
file can be found in \qcode{/etc/hostname}.

%% configuration_file
\TheProperty[hostname:configuration_file]{configuration\_\\file}
\TheProperty*[hostname!configuration\_file]{configuration\_file \Arg{name}}

The file \Arg{name} of the host name configuration file. The configuration file
is places inside the specified configuration directory.
For the distribution
\begin{inparaitem}
\item[\TheDistribution{ubuntu}] Ubuntu 10.04,
\item[\TheDistribution{ubuntu}] Ubuntu 12.04
\end{inparaitem}
the value is set to the file \qcode{hostname}. The hostname configuration
file is normally found in the directory \qcode{/etc}.

%% packages
\TheProperty[hostname:packages]{packages}
\TheProperty*[hostname!packages]{packages \Arg{packages}}

The list of \Arg{packages} needed for the hostname service. The packages
are installed with the specified installation command.
For the distribution
\begin{inparaitem}
\item[\TheDistribution{ubuntu}] Ubuntu 10.04,
\item[\TheDistribution{ubuntu}] Ubuntu 12.04
\end{inparaitem}
the value is set to the package \qcode{hostname}.

%% restart_services
\TheProperty[hostname:restart_services]{restart\_\\services}
\TheProperty*[hostname!mysql!restart\_services]{restart\_services \Arg{services}}

The list of \Arg{services} to restart the hostname service.
For the distribution
\begin{inparaitem}
\item[\TheDistribution{ubuntu}] Ubuntu 10.04
\end{inparaitem}
the value is set to \qcode{}, empty, i.e. no services.

%% charset
\TheProperty[hostname:charset]{charset}
\TheProperty*[hostname!charset]{charset \Arg{name}}

Sets the default character set \Arg{name} in which configuration files are 
saved. For the distribution
\begin{inparaitem}
\item[\TheDistribution{ubuntu}] Ubuntu 10.04,
\item[\TheDistribution{ubuntu}] Ubuntu 12.04
\end{inparaitem}
the value is set to \qcode{utf-8}.
