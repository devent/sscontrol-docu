\subsection{Hostname Service}

\begin{multicols}{2}

Der Hostname wird von verschiedenen Services verwenden um den Namen der Maschine
zu erhalten.

\paragraph{Hostname Profil}

Das Hostname Profil wird eingeleitet durch die \directive{hostname}. Gefolgt von
den folgenden Direktiven:

\begin{asparadesc}
\item[\directive{configuration\_directory}] beschreibt in welchem Verzeichnis
sich die Konfiguration für den Hostnamen befinden. Unter Ubuntu und Debian
Systemen wird die Konfiguration in der Datei \code{/etc/hostname} gespeichert,
somit ist das Verzeichnis \code{/etc}.
\item[\directive{restart\_command}] enthält das Kommando mit dem man im
laufendem Betrieb den Hostnamen ändern kann.  Unter Ubuntu und Debian
Systemen ist es das Kommando\\
\code{/etc/init.d/hostname restart}.
\item[\directive{status\_command}] enthält das Kommando mit dem man den
aktuellen Hostnamen abfragen kann. Unter Ubuntu und Debian Systemen ist es das
Kommando \code{/bin/hostname}.
\end{asparadesc}

\end{multicols}

