\subsection{Hostname - Ubuntu 10.04}

Ubuntu 10.04 comes with hostname version 3.03 in the main repository.
The properties below have already set meaningful default values.

\subsubsection*{System}

\TheProperty[hostname:install_command]{install\_\\command}
\TheProperty*[hostname!install\_command]{install\_command \TypeString}

The command to install package on the Ubuntu system.
Default value set to \qcode{/usr/bin/aptitude update \&\& /usr/bin/aptitude install}.

\subsubsection*{Hostname}

\TheProperty[hostname:restart_command]{restart\_\\command}
\TheProperty*[hostname!restart\_command]{restart\_command \TypeString}

The command to reload the hostname.
Default value set to \qcode{/sbin/restart hostname}

\TheProperty[hostname:configuration_directory]{configuration\_\\directory}
\TheProperty*[hostname!configuration\_directory]{configuration\_directory \TypeString}

The path of the hostname configuration directory.
Default value set to \qcode{/etc}, so that the hostname configuration
file can be found in \qcode{/etc/hostname}.

\TheProperty[hostname:configuration_file]{configuration\_\\file}
\TheProperty*[hostname!configuration\_file]{configuration\_file \TypeString}

The file name of the {@code hostname} configuration file.
Default value set to the file \qcode{hostname}. The hostname configuration
file is normally found in \qcode{/etc}.

\TheProperty[hostname:packages]{packages}
\TheProperty*[hostname!packages]{packages \TypeList}

The packages needed for the hostname service.
Default value set to the package \qcode{hostname}.

\begin{lstlisting}[style=Sscontrol,
label={lst:hostname_ubuntu_profile},
title={Ubuntu hostname profile.}]
profile "ubuntu_10_04", {
    system {
        install_command "/usr/bin/aptitude update && /usr/bin/aptitude install"
    }
    hostname {
        configuration_directory "/etc"
        restart_command "/sbin/restart hostname"
    }
}
\end{lstlisting}
