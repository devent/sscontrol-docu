\label{sec:postfix_profile}
\subsection{Postfix Profile}

%% storage
\TheProperty[mail:postfix:storage]{storage}
\TheProperty*[mail!postfix!storage]{storage \Arg{name}}

The \Arg{name} of the postfix storage. Can be one of the following 
storage types.
\begin{asparaitem}
%
\item[\qcode{hash}:] 
the normal postfix storage. The aliases and domains are stored in Hash files.
Supports shared domains, UNIX system accounts;
separate domains, Unix system accounts; separate domains, non-UNIX accounts.
%
\item[\qcode{postfix}:] supports the same types as the normal 
postfix storage but with MySQL as the back-end mail.
%
\end{asparaitem}

%% delivery
\TheProperty[mail:postfix:delivery]{delivery}
\TheProperty*[mail!postfix!delivery]{delivery \Arg{name}}

The \Arg{name} of the mail delivery. Can be one of the following 
delivery types.
\begin{asparaitem}
%
\item[\qcode{courier}:] 
Courier/POP/IMAP mail delivery.
%
\end{asparaitem}

%% banner
\TheProperty[mail:postfix:banner]{banner}
\TheProperty*[mail!postfix!banner]{banner \Arg{text}}

Sets the banner \Arg{text} that follows the 220 status code in the SMTP 
greeting banner. The text can contain storage specific variables. Caution
must be taked with banner text that contains backslashes \qcode{\textbackslash} and 
dollar signs \qcode{\$}. Dollar signs may be treated as references to 
captured subsequences and backslashes are used to escape literal characters 
in the banner text.
For the distribution
\begin{inparaitem}
\item[\TheDistribution{ubuntu}] Ubuntu 10.04
\end{inparaitem}
the value is set to \qcode{\textbackslash\$myhostname ESMTP \textbackslash\$mail\_name}.

%% unknown_local_recipient_reject_code
\TheProperty[mail:postfix:unknown_local_recipient_reject_code]{unknown\_local\_\\recipient\_\\reject\_code}
\TheProperty*[mail!postfix!unknown\_local\_recipient\_reject\_code]{unknown\_local\_recipient\_reject\_code \Arg{code}}

The numerical Postfix/SMTP server response code when a local recipient 
address is unknown.
For the distribution
\begin{inparaitem}
\item[\TheDistribution{ubuntu}] Ubuntu 10.04
\end{inparaitem}
the value is set to \code{550}.

%% default_destinations
\TheProperty[mail:postfix:default_destinations]{default\_\\destinations}
\TheProperty*[mail!postfix!default\_destinations]{default\_destinations \Arg{domains}}

List of \Arg{domains} that are appended to the destinations.
Caution must be taked with the domain names that contains backslashes \qcode{\textbackslash} and 
dollar signs \qcode{\$}. Dollar signs may be treated as references to 
captured subsequences and backslashes are used to escape literal characters  in the domain name.
The value is set to \qcode{\textbackslash\$myhostname, localhost.\textbackslash\$mydomain}.

%% mailbox_base_pattern
\TheProperty[mail:postfix:mailbox_pattern]{mailbox\_pattern}
\TheProperty*[mail!postfix!mailbox\_pattern]{mailbox\_pattern \Arg{pattern}}

Sets the \Arg{pattern} for the directories that are created for each virtual
domain and virtual user under the mailbox base directory. 
Various variables can be used to specify the pattern.
Each of the variables needs to be eclosed in \qcode{<>}.
\begin{compactitem}
\item[\qcode{domain}:] the domain name of the virtual user;
\item[\qcode{user}:] the user name of the virtual user;
\end{compactitem}
Default is set the \qcode{<domain>/<user>} pattern.

%% delay_warning_time
\TheProperty[mail:postfix:delay_warning_time]{delay\_warning\_\\time}
\TheProperty*[mail!postfix!delay\_warning\_time]{delay\_warning\_time \Arg{time}}

Sets the \Arg{time} after which the sender receives a copy of the message
headers of the mails that are still queued. The time is expected to be in 
the ISO 8601 format for durations.
For the distribution
\begin{inparaitem}
\item[\TheDistribution{ubuntu}] Ubuntu 10.04
\end{inparaitem}
the value is set to four hours \qcode{PT4H}.

%% maximal_queue_lifetime
\TheProperty[mail:postfix:maximal_queue_lifetime]{maximal\_\\queue\_lifetime}
\TheProperty*[mail!postfix!maximal\_queue\_lifetime]{maximal\_queue\_lifetime \Arg{time}}

Sets the \Arg{time} after which the message is consider as not possible to deliver. 
The time is expected to be in the ISO 8601 format for durations.
For the distribution
\begin{inparaitem}
\item[\TheDistribution{ubuntu}] Ubuntu 10.04
\end{inparaitem}
the value is set to seven days \qcode{P7D}.

%% minimal_retries_delay
\TheProperty[mail:postfix:minimal_retries_delay]{minimal\_\\retries\_delay}
\TheProperty*[mail!postfix!minimal\_retries\_delay]{minimal\_retries\_delay \Arg{time}}

Sets the minimal \Arg{time} between attempts to deliver a deferred message.
The time is expected to be in the ISO 8601 format for durations.
For the distribution
\begin{inparaitem}
\item[\TheDistribution{ubuntu}] Ubuntu 10.04
\end{inparaitem}
the value is set to fifteen minutes \qcode{PT15M}.

%% maximal_retries_delay
\TheProperty[mail:postfix:maximal_retries_delay]{maximal\_\\retries\_delay}
\TheProperty*[mail!postfix!maximal\_retries\_delay]{maximal\_retries\_delay \Arg{time}}

Sets the maximal \Arg{time} between attempts to deliver a deferred message.
The time is expected to be in the ISO 8601 format for durations.
For the distribution
\begin{inparaitem}
\item[\TheDistribution{ubuntu}] Ubuntu 10.04
\end{inparaitem}
the value is set to two hours \qcode{PT2H}.

%% helo_timeout
\TheProperty[mail:postfix:helo_timeout]{helo\_timeout}
\TheProperty*[mail!postfix!helo\_timeout]{helo\_timeout \Arg{time}}

Sets the Postfix/SMTP client \Arg{time} limit for sending the HELO or
EHLO command, and for receiving the initial remote SMTP server response.
The time is expected to be in the ISO 8601 format for durations.
For the distribution
\begin{inparaitem}
\item[\TheDistribution{ubuntu}] Ubuntu 10.04
\end{inparaitem}
the value is set to one minute \qcode{PT1M}.

%% recipient_limit
\TheProperty[mail:postfix:recipient_limit]{recipient\_limit}
\TheProperty*[mail!postfix!recipient\_limit]{recipient\_limit \Arg{number}}

Sets the maximal \Arg{number} of recipients that the Postfix/SMTP server
accepts per message delivery request.
For the distribution
\begin{inparaitem}
\item[\TheDistribution{ubuntu}] Ubuntu 10.04
\end{inparaitem}
the value is set to sixteen \code{16}.

%% back_off_error_limit
\TheProperty[mail:postfix:back_off_error_limit]{back\_off\_\\error\_limit}
\TheProperty*[mail!postfix!back\_off\_error\_limit]{back\_off\_error\_limit \Arg{number}}

Sets the \Arg{number} of errors before the Postfix/SMTP server slows
down all its responses.
For the distribution
\begin{inparaitem}
\item[\TheDistribution{ubuntu}] Ubuntu 10.04
\end{inparaitem}
the value is set to three \code{3}.

%% blocking_error_limit
\TheProperty[mail:postfix:blocking_error_limit]{blocking\_\\error\_limit}
\TheProperty*[mail!postfix!blocking\_error\_limit]{blocking\_error\_limit \Arg{number}}

Sets the maximal \Arg{number} of errors a remote SMTP client is allowed
to make without delivering mail.
For the distribution
\begin{inparaitem}
\item[\TheDistribution{ubuntu}] Ubuntu 10.04
\end{inparaitem}
the value is set to twelve \code{12}.

%% helo_restrictions
\TheProperty[mail:postfix:helo_restrictions]{helo\_\\restrictions}
\TheProperty*[mail!postfix!helo\_restrictions]{helo\_restrictions \Arg{restrictions}}

Sets the \Arg{restrictions} that the Postfix/SMTP server applies in the
context of a client HELO command.
For the distribution
\begin{inparaitem}
\item[\TheDistribution{ubuntu}] Ubuntu 10.04
\end{inparaitem}
the restrictions are
\begin{compactitem}
\item \qcode{permit\_mynetworks},
\item \qcode{warn\_if\_reject reject\_non\_fqdn\_hostname},
\item \qcode{reject\_invalid\_hostname},
\item \qcode{permit}.
\end{compactitem}

%% sender_restrictions
\TheProperty[mail:postfix:sender_restrictions]{sender\_\\restrictions}
\TheProperty*[mail!postfix!sender\_restrictions]{sender\_restrictions \Arg{restrictions}}

Sets the \Arg{restrictions} that the Postfix/SMTP server applies in the
context of a client MAIL FROM command.
For the distribution
\begin{inparaitem}
\item[\TheDistribution{ubuntu}] Ubuntu 10.04
\end{inparaitem}
the restrictions are
\begin{compactitem}
\item \qcode{permit\_mynetworks},
\item \qcode{warn\_if\_reject reject\_non\_fqdn\_sender},
\item \qcode{reject\_unknown\_sender\_domain},
\item \qcode{reject\_unauth\_pipelining},
\item \qcode{permit},
\end{compactitem}

%% client_restrictions
\TheProperty[mail:postfix:client_restrictions]{client\_\\restrictions}
\TheProperty*[mail!postfix!client\_restrictions]{client\_restrictions \Arg{restrictions}}

Sets the \Arg{restrictions} that the Postfix/SMTP server applies in the
context of a client connection request.
For the distribution
\begin{inparaitem}
\item[\TheDistribution{ubuntu}] Ubuntu 10.04
\end{inparaitem}
the restrictions are
\begin{compactitem}
\item \qcode{reject\_rbl\_client sbl.spamhaus.org},
\item \qcode{reject\_rbl\_client blackholes.easynet.nl},
\item \qcode{reject\_rbl\_client dnsbl.njabl.org}.
\end{compactitem}

%% recipient_restrictions
\TheProperty[mail:postfix:recipient_restrictions]{recipient\_\\restrictions}
\TheProperty*[mail!postfix!recipient\_restrictions]{recipient\_restrictions \Arg{restrictions}}

For the distribution
\begin{inparaitem}
\item[\TheDistribution{ubuntu}] Ubuntu 10.04
\end{inparaitem}
the restrictions are
\begin{compactitem}
\item \qcode{reject\_unauth\_pipelining},
\item \qcode{permit\_mynetworks},
\item \qcode{reject\_non\_fqdn\_recipient},
\item \qcode{reject\_unknown\_recipient\_domain},
\item \qcode{reject\_unauth\_destination},
\item \qcode{permit}.
\end{compactitem}

%% data_restrictions
\TheProperty[mail:postfix:data_restrictions]{data\_\\restrictions}
\TheProperty*[mail!postfix!data\_restrictions]{data\_restrictions \Arg{restrictions}}

Sets the \Arg{restrictions} that the Postfix/SMTP server applies in the
context of a client RCPT TO command.
For the distribution
\begin{inparaitem}
\item[\TheDistribution{ubuntu}] Ubuntu 10.04
\end{inparaitem}
the restrictions are
\begin{compactitem}
\item \qcode{reject\_unauth\_pipelining}.
\end{compactitem}

%% helo_required
\TheProperty[mail:postfix:helo_required]{helo\_required}
\TheProperty*[mail!postfix!helo\_required]{helo\_required \Arg{flag}}

Sets requirement that a remote SMTP client introduces itself with the HELO/EHLO
command. The flag is expected to be either \qcode{true} or \qcode{false}.
For the distribution
\begin{inparaitem}
\item[\TheDistribution{ubuntu}] Ubuntu 10.04
\end{inparaitem}
the flag is set to \qcode{true}.

%% delay_reject
\TheProperty[mail:postfix:delay_reject]{delay\_reject}
\TheProperty*[mail!postfix!delay\_reject]{delay\_reject \Arg{flag}}

Sets requirement to wait until the RCPT TO/ETRN command. 
The flag is expected to be either \qcode{true} or \qcode{false}.
For the distribution
\begin{inparaitem}
\item[\TheDistribution{ubuntu}] Ubuntu 10.04
\end{inparaitem}
the flag is set to \qcode{true}.

%% disable_vrfy_command
\TheProperty[mail:postfix:disable_vrfy_command]{disable\_\\vrfy\_command}
\TheProperty*[mail!postfix!disable\_vrfy\_command]{disable\_vrfy\_command \Arg{flag}}

Sets to disable the SMTP VRFY command.
The flag is expected to be either \qcode{true} or \qcode{false}.
For the distribution
\begin{inparaitem}
\item[\TheDistribution{ubuntu}] Ubuntu 10.04
\end{inparaitem}
the flag is set to \qcode{true}.
