\section{Profile}

Die Profile beschreiben die Umgebungen in denen die
Services einrichten soll. Ein typisches Profil enthält einen Namen,
wie ``ubuntu'', und eine Auflistung
der Services, die eingerichtet werden sollen. Nur die Services die in einem
Profil gelistet sind werden verwaltet.

Die Services werden in der Reihenfolge abgearbeitet, die sie im Profil stehen.
Nur wenn ein Service erfolgreich abgeschlossen ist, wird das nächste Service eingerichtet.

\subsection{Profile Direktiven}

Ein Profilscript wird eingeleitet durch die \directive{profiles}.
Danach wird ein neues Profil angelegt durch die
\directive{create\_profile(name)}. Der \parameter{name} ist eine Zeichenfolge und
bezeichnet das Profil. Das Profil kann durch diesen Namen ausgewählt werden. 
Danach kommt eine Auflistung der
Services, die verwalten werden sollen.

Wie man am Beispiel sehen kann enthalten die Profildirektiven Anweisungen
über die Ubuntu-Umgebung, also wo sich die Konfiguration befindet, welche
Kommandos man für die Services braucht, usw.

\begin{lstlisting}[style=Java, caption=Beispiel Profil eines Ubuntu Servers]
profiles {
	create_profile("ubuntu") {
		system {
			packages_install_command 'export DEBIAN_FRONTEND=noninteractive; /usr/bin/apt-get -q -y update && /usr/bin/apt-get -q -y install'
			tmp_directory '/tmp'
			wget_command '/usr/bin/wget -nv'
			targz_command '/bin/tar xf'
			local_bin_directory '/usr/local/bin'
			chown_command '/bin/chown -R'
			chmod_command '/bin/chmod -R'
		}
		hostname {
			configuration_directory '/etc'
			restart_command '/etc/init.d/hostname restart'
			status_command '/bin/hostname'
		}
		hosts {
			configuration_directory '/etc'
			status_command '/bin/cat /etc/hosts'
		}
		database {
			configuration_directory '/etc/mysql/conf.d'
			restart_command '/sbin/restart mysql'
			status_command '/sbin/status mysql'
			
			packages 'mysql-server, mysql-client'
			
			mysql_admin_command '/usr/bin/mysqladmin'
			mysql_command '/usr/bin/mysql'
			mysql_status_pattern 'Uptime.*'
			
		}
		httpd {
			configuration_directory '/etc/apache2'
			restart_command '/etc/init.d/apache2 restart'
			status_command '/usr/sbin/apache2ctl configtest 1> echo'
			status_command_expected ''
			
			domain_web_directory_prefix '/var/www/$domain_name$/web'
			domain_ssl_directory_prefix '/var/www/$domain_name$/ssl'
			domain_page_directory_prefix '/var/www/$domain_name$/$page_name$/web'
			
			packages 'apache2'
		}
	}
}
\end{lstlisting}

