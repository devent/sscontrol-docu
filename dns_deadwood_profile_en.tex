\label{sec:deadwood_profile}
\subsection{Deadwood Profile}

The Deadwood service comes in the distribution Ubuntu 14.04 in version 2.0.09 
in the universe repository.

The Deadwood service is only a recursive DNS server, the 
DNS \Statement*[dns:record]{record} statements are going to be ignored.

%% install_command
\TheProperty[dns:deadwood:install_command]{install\_\\command}
\TheProperty*[dns!deadwood!install\_command]{install\_command \Arg{command}}

The \Arg{command} to install package on the system. Can be a full path or
just the command name that can be found in the current search path. 
For the distribution
\begin{inparaitem}
\item[\TheDistribution{ubuntu}] Ubuntu 14.04
\end{inparaitem}
the value is set to \qcode{/usr/bin/aptitude}.

%% deadwood_restart_command
%% restart_command
\TheProperty[dns:deadwood:restart_command]{restart\_\\command}
\TheProperty*[dns!deadwood!restart\_command]{restart\_command \Arg{command}}\\
\TheProperty*[dns!deadwood!deadwood\_restart\_command]{deadwood\_restart\_command \Arg{command}}

The \Arg{command} to restart the service. Can be a full path or
just the command name that can be found in the current search path. 
The with \qcode{deadwood\_} prefixed property have precendence over the 
nonprefixed property.
For the distribution
\begin{inparaitem}
\item[\TheDistribution{ubuntu}] Ubuntu 14.04
\end{inparaitem}
the value is set to \qcode{/etc/init.d/deadwood}.

%% deadwood_status_command
%% status_command
\TheProperty[dns:deadwood:status_command]{status\_\\command}
\TheProperty*[dns!deadwood!status\_command]{status\_command \Arg{command}}\\
\TheProperty*[dns!deadwood!deadwood\_status\_command]{deadwood\_status\_command \Arg{command}}

The \Arg{command} to request the status the service. Can be a full path or
just the command name that can be found in the current search path. 
The with \qcode{deadwood\_} prefixed property have precendence over the 
nonprefixed property.
For the distribution
\begin{inparaitem}
\item[\TheDistribution{ubuntu}] Ubuntu 14.04
\end{inparaitem}
the value is set to \qcode{/etc/init.d/deadwood}.

%% group_add_command
\TheProperty[dns:deadwood:group_add_command]{group\_add\_\\command}
\TheProperty*[dns!deadwood!group\_add\_command]{group\_add\_command \Arg{command}}\\

The \Arg{command} to add a new local group. Can be a full path or
just the command name that can be found in the current search path. 
For the distribution
\begin{inparaitem}
\item[\TheDistribution{ubuntu}] Ubuntu 14.04
\end{inparaitem}
the value is set to \qcode{/usr/sbin/groupadd}.

%% user_add_command
\TheProperty[dns:deadwood:user_add_command]{user\_add\_\\command}
\TheProperty*[dns!deadwood!user\_add\_command]{user\_add\_command \Arg{command}}\\

The \Arg{command} to add a new local user. Can be a full path or
just the command name that can be found in the current search path. 
For the distribution
\begin{inparaitem}
\item[\TheDistribution{ubuntu}] Ubuntu 14.04
\end{inparaitem}
the value is set to \qcode{/usr/sbin/useradd}.

%% id_command
\TheProperty[dns:deadwood:id_command]{id\_command}
\TheProperty*[dns!deadwood!id\_command]{id\_command \Arg{command}}\\

The \Arg{command} to reguests information about local user and local group. Can 
be a full path or just the command name that can be found in the current search path. 
For the distribution
\begin{inparaitem}
\item[\TheDistribution{ubuntu}] Ubuntu 14.04
\end{inparaitem}
the value is set to \qcode{/usr/bin/id}.

%% chmod_command
\TheProperty[dns:deadwood:chmod_command]{chmod\_command}
\TheProperty*[dns!deadwood!chmod\_command]{chmod\_command \Arg{command}}\\

The \Arg{command} to change the file mode bits. Can be a full path or
just the command name that can be found in the current search path. 
For the distribution
\begin{inparaitem}
\item[\TheDistribution{ubuntu}] Ubuntu 14.04
\end{inparaitem}
the value is set to \qcode{/bin/chmod}.

%% chown_command
\TheProperty[dns:deadwood:chown_command]{chown\_command}
\TheProperty*[dns!deadwood!chown\_command]{chown\_command \Arg{command}}\\

The \Arg{command} to change the file owner. Can be a full path or
just the command name that can be found in the current search path. 
For the distribution
\begin{inparaitem}
\item[\TheDistribution{ubuntu}] Ubuntu 14.04
\end{inparaitem}
the value is set to \qcode{/bin/chown}.

%% deadwood_command
\TheProperty[dns:deadwood:deadwood_command]{deadwood\_\\command}
\TheProperty*[dns!deadwood!deadwood\_command]{deadwood\_command \Arg{command}}\\

The \Arg{command} of the Deadwood service. Can be a full path or
just the command name that can be found in the current search path. 
For the distribution
\begin{inparaitem}
\item[\TheDistribution{ubuntu}] Ubuntu 14.04
\end{inparaitem}
the value is set to \qcode{/usr/sbin/deadwood}.

%% update_rc_command
\TheProperty[dns:deadwood:update_rc_command]{update\_rc\_\\command}
\TheProperty*[dns!deadwood!update\_rc\_command]{update\_rc\_command \Arg{command}}\\

The \Arg{command} to update the System V init type scripts. Can be a full path or
just the command name that can be found in the current search path. 
For the distribution
\begin{inparaitem}
\item[\TheDistribution{ubuntu}] Ubuntu 14.04
\end{inparaitem}
the value is set to \qcode{/usr/sbin/update-rc.d}.

%% groups_file
\TheProperty[dns:deadwood:groups_file]{groups\_file}
\TheProperty*[dns!deadwood!groups\_file]{groups\_file \Arg{path}}\\

The \Arg{path} of the local groups file.
For the distribution
\begin{inparaitem}
\item[\TheDistribution{ubuntu}] Ubuntu 14.04
\end{inparaitem}
the value is set to \qcode{/etc/group}.

%% users_file
\TheProperty[dns:deadwood:users_file]{users\_file}
\TheProperty*[dns!deadwood!users\_file]{users\_file \Arg{path}}\\

The \Arg{path} of the local users file.
For the distribution
\begin{inparaitem}
\item[\TheDistribution{ubuntu}] Ubuntu 14.04
\end{inparaitem}
the value is set to \qcode{/etc/passwd}.

%% deadwood_configuration_directory
%% configuration_directory
\TheProperty[dns:deadwood:configuration_directory]{configuration\_\\directory}
\TheProperty*[dns!deadwood!configuration\_directory]{configuration\_directory \Arg{path}}\\
\TheProperty*[dns!deadwood!deadwood\_configuration\_directory]{deadwood\_configuration\_directory \Arg{path}}

The \Arg{path} of the service configuration directory.
The with \qcode{deadwood\_} prefixed property have precendence over the 
nonprefixed property.
For the distribution
\begin{inparaitem}
\item[\TheDistribution{ubuntu}] Ubuntu 14.04
\end{inparaitem}
the value is set to \qcode{/etc/maradns/deadwood}.

%% deadwood_configuration_file
%% configuration_file
\TheProperty[dns:deadwood:configuration_file]{configuration\_\\file}
\TheProperty*[dns!deadwood!configuration\_file]{configuration\_file \Arg{name}}\\
\TheProperty*[dns!deadwood!deadwood\_configuration\_file]{deadwood\_configuration\_file \Arg{name}}

The \Arg{name} of the service configuration file. The file is saved
under the configuration directory. The with \qcode{deadwood\_} prefixed 
property have precendence over the nonprefixed property.
For the distribution
\begin{inparaitem}
\item[\TheDistribution{ubuntu}] Ubuntu 14.04
\end{inparaitem}
the value is set to \qcode{dwood3rc}.

%% deadwood_script_file
\TheProperty[dns:deadwood:deadwood_script_file]{deadwood\_\\script\_\\file}
\TheProperty*[dns!deadwood!deadwood\_script\_file]{deadwood\_script\_file \Arg{path}}\\

The \Arg{path} of the service service init script file.
For the distribution
\begin{inparaitem}
\item[\TheDistribution{ubuntu}] Ubuntu 14.04
\end{inparaitem}
the value is set to \qcode{/etc/init.d/deadwood}.

%% system_name
\TheProperty[dns:deadwood:system_name]{system\_name}
\TheProperty*[dns!deadwood!system\_name]{system\_name \Arg{name}}

The system \Arg{name}. 
For the distribution
\begin{inparaitem}
\item[\TheDistribution{ubuntu}] Ubuntu 14.04
\end{inparaitem}
the value is set to \qcode{ubuntu}.

%% packages
\TheProperty[dns:deadwood:packages]{packages}
\TheProperty*[dns!deadwood!packages]{packages \Arg{packages}}

The list of \Arg{packages} that are needed for the service.
For the distribution
\begin{inparaitem}
\item[\TheDistribution{ubuntu}] Ubuntu 14.04
\end{inparaitem}
the value is set to \qcode{maradns-deadwood, dpkg}.

%% restart_services
\TheProperty[dns:deadwood:restart_services]{restart\_\\services}
\TheProperty*[dns!deadwood!restart\_services]{restart\_services \Arg{services}}\\
\TheProperty*[dns!deadwood!deadwood\_restart\_services]{deadwood\_restart\_services \Arg{services}}

The list of \Arg{services} that are needed to be restarted after the 
configuration was deployed to the server. The with \qcode{deadwood\_} prefixed 
property have precendence over the nonprefixed property.
For the distribution
\begin{inparaitem}
\item[\TheDistribution{ubuntu}] Ubuntu 14.04
\end{inparaitem}
the value is set to \qcode{}, empty, i.e. no services.

%% restart_flags
\TheProperty[dns:deadwood:restart_flags]{restart\_flags}
\TheProperty*[dns!deadwood!restart\_flags]{restart\_flags \Arg{flags}}\\
\TheProperty*[dns!deadwood!deadwood\_restart\_flags]{deadwood\_restart\_flags \Arg{flags}}

The \Arg{flags} for the restart command. The with \qcode{deadwood\_} prefixed 
property have precendence over the nonprefixed property.
For the distribution
\begin{inparaitem}
\item[\TheDistribution{ubuntu}] Ubuntu 14.04
\end{inparaitem}
the value is set to \qcode{restart}.

%% status_flags
\TheProperty[dns:deadwood:status_flags]{status\_flags}
\TheProperty*[dns!deadwood!status\_flags]{status\_flags \Arg{flags}}\\
\TheProperty*[dns!deadwood!deadwood\_status\_flags]{deadwood\_status\_flags \Arg{flags}}

The \Arg{flags} for the status command. The with \qcode{deadwood\_} prefixed 
property have precendence over the nonprefixed property.
For the distribution
\begin{inparaitem}
\item[\TheDistribution{ubuntu}] Ubuntu 14.04
\end{inparaitem}
the value is set to \qcode{status}.

%% charset
\TheProperty[dns:deadwood:charset]{charset}
\TheProperty*[dns!deadwood!charset]{charset \Arg{name}}

The character set \Arg{name} of the service configuration files. 
For the distribution
\begin{inparaitem}
\item[\TheDistribution{ubuntu}] Ubuntu 14.04
\end{inparaitem}
the value is set to \qcode{utf-8}.

%% deadwood_user
\TheProperty[dns:deadwood:deadwood_user]{deadwood\_user}
\TheProperty*[dns!deadwood!deadwood\_user]{deadwood\_user \Arg{name}}

The local user \Arg{name} under which the service is run.
For the distribution
\begin{inparaitem}
\item[\TheDistribution{ubuntu}] Ubuntu 14.04
\end{inparaitem}
the value is set to \qcode{deadwood}.

%% deadwood_group
\TheProperty[dns:deadwood:deadwood_group]{deadwood\_group}
\TheProperty*[dns!deadwood!deadwood\_group]{deadwood\_group \Arg{name}}

Sets local group \Arg{name} under which the service is run.
For the distribution
\begin{inparaitem}
\item[\TheDistribution{ubuntu}] Ubuntu 14.04
\end{inparaitem}
the value is set to \qcode{deadwood}.

%% ipv4_pattern
\TheProperty[dns:deadwood:ipv4_pattern]{ipv4\_pattern}
\TheProperty*[dns!deadwood!ipv4\_pattern]{ipv4\_pattern \Arg{pattern}}

The IPv4 match \Arg{pattern}.
For the distribution
\begin{inparaitem}
\item[\TheDistribution{ubuntu}] Ubuntu 14.04
\end{inparaitem}
the value is set to \qcode{\textasciicircum((\textbackslash\textbackslash{d}\{1,3\}\textbackslash\textbackslash{.})\{3\}\textbackslash\textbackslash{d}\{1,3\})\$}.

%% ipv6_pattern
\TheProperty[dns:deadwood:ipv6_pattern]{ipv6\_pattern}
\TheProperty*[dns!deadwood!ipv6\_pattern]{ipv6\_pattern \Arg{pattern}}

The IPv6 match \Arg{pattern}.
For the distribution
\begin{inparaitem}
\item[\TheDistribution{ubuntu}] Ubuntu 14.04
\end{inparaitem}
the value is set to \qcode{\textasciicircum(([0-9a-fA-F]\{0,4\}:)\{1,7\}[0-9a-fA-F])\{0,4\}\$}.

%% deadwood_max_requests
\TheProperty[dns:deadwood:deadwood_max_requests]{deadwood\_max\_\\requests}
\TheProperty*[dns!deadwood!deadwood\_max\_requests]{deadwood\_max\_requests \Arg{requests}}

The the maximum \Arg{requests}.
For the distribution
\begin{inparaitem}
\item[\TheDistribution{ubuntu}] Ubuntu 14.04
\end{inparaitem}
the value is set to \code{8}.

%% deadwood_handle_overload
\TheProperty[dns:deadwood:deadwood_handle_overload]{deadwood\_\\handle\_\\overload}
\TheProperty*[dns!deadwood!deadwood\_handle\_overload]{deadwood\_handle\_overload \Arg{enabled}}

The handle overload \Arg{enabled}.
For the distribution
\begin{inparaitem}
\item[\TheDistribution{ubuntu}] Ubuntu 14.04
\end{inparaitem}
the value is set to \code{true}.

%% deadwood_max_cache
\TheProperty[dns:deadwood:deadwood_max_cache]{deadwood\_max\_\\cache}
\TheProperty*[dns!deadwood!deadwood\_max\_cache]{deadwood\_max\_cache \Arg{size}}

The maximum cache \Arg{size}.
For the distribution
\begin{inparaitem}
\item[\TheDistribution{ubuntu}] Ubuntu 14.04
\end{inparaitem}
the value is set to \code{60000}.

%% deadwood_cache_file
\TheProperty[dns:deadwood:deadwood_cache_file]{deadwood\_\\cache\_\\file}
\TheProperty*[dns!deadwood!deadwood\_cache\_file]{deadwood\_cache\_file \Arg{name}}

The cache file \Arg{name}.
For the distribution
\begin{inparaitem}
\item[\TheDistribution{ubuntu}] Ubuntu 14.04
\end{inparaitem}
the value is set to \qcode{dw\_cache}.

%% deadwood_resurrections
\TheProperty[dns:deadwood:deadwood_resurrections]{deadwood\_\\resurrections}
\TheProperty*[dns!deadwood!deadwood\_resurrections]{deadwood\_resurrections \Arg{enabled}}

The fetch expired records \Arg{enabled}.
For the distribution
\begin{inparaitem}
\item[\TheDistribution{ubuntu}] Ubuntu 14.04
\end{inparaitem}
the value is set to \code{true}.

%% deadwood_filter_rfc1918
\TheProperty[dns:deadwood:deadwood_filter_rfc1918]{deadwood\_\\filter\_\\rfc1918}
\TheProperty*[dns!deadwood!deadwood\_filter\_rfc1918]{deadwood\_filter\_rfc1918 \Arg{enabled}}

The filter RFC1928 \Arg{enabled}.
For the distribution
\begin{inparaitem}
\item[\TheDistribution{ubuntu}] Ubuntu 14.04
\end{inparaitem}
the value is set to \code{true}.

%% deadwood_reject_mx
\TheProperty[dns:deadwood:deadwood_reject_mx]{deadwood\_\\reject\_\\mx}
\TheProperty*[dns!deadwood!deadwood\_reject\_mx]{deadwood\_reject\_mx \Arg{enabled}}

The reject MX queries \Arg{enabled}.
For the distribution
\begin{inparaitem}
\item[\TheDistribution{ubuntu}] Ubuntu 14.04
\end{inparaitem}
the value is set to \code{true}.

%% default_binding_address
\TheProperty[dns:deadwood:default_binding_address]{default\_\\binding\_\\address}
\TheProperty*[dns!deadwood!default\_binding\_address]{default\_binding\_address \Arg{addresses}}

Sets the default binding \Arg{address}. For the distribution
\begin{inparaitem}
\item[\TheDistribution{ubuntu}] Ubuntu 14.04
\end{inparaitem}
the value is set to \qcode{127.0.0.1}.

%% default_binding_port
\TheProperty[dns:deadwood:default_binding_port]{default\_\\binding\_\\port}
\TheProperty*[dns!deadwood!default\_binding\_port]{default\_binding\_port \Arg{port}}

Sets the default binding \Arg{port}. For the distribution
\begin{inparaitem}
\item[\TheDistribution{ubuntu}] Ubuntu 14.04
\end{inparaitem}
the value is set to \qcode{53}.

%% servers_group_icann
\TheProperty[dns:deadwood:servers_group_icann]{servers\_group\_\\icann}
\TheProperty*[dns!deadwood!servers\_group\_icann]{servers\_group\_icann \Arg{addresses}}

Sets the ICANN root server \Arg{addresses}. Those server groups are used in 
the DNS service script when used in 
the \Statement*[dns:servers:root]{servers-root} statement.
For the distribution
\begin{inparaitem}
\item[\TheDistribution{ubuntu}] Ubuntu 14.04
\end{inparaitem}
the value is set to \\
\qcode{198.41.0.4, 192.228.79.201, 192.33.4.12, 199.7.91.13, 192.203.230.10,\\
192.5.5.241, 192.112.36.4, 128.63.2.53, 192.36.148.17, 192.58.128.30, \\
193.0.14.129, 199.7.83.42, 202.12.27.33}.

