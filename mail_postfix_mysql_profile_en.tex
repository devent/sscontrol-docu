\label{sec:postfix_mysql_profile}
\subsection[Postfix MySQL Profile]{MySQL Storage Profile}

%% install_command
\TheProperty[mail:postfix:mysql:install_command]{install\_\\command}
\TheProperty*[mail!postfix!mysql!install\_command]{install\_command \Arg{command}}

The \Arg{command} to install package on the system. Can be a full path or
just the command name that can be found in the current search path. 
For the distribution
\begin{inparaitem}
\item[\TheDistribution{ubuntu}] Ubuntu 10.04
\end{inparaitem}
the value is set to \qcode{/usr/bin/aptitude update \&\& DEBIAN\_FRONTEND=noninteractive /usr/bin/aptitude -y install}.

%% restart_command
\TheProperty[mail:postfix:mysql:restart_command]{restart\_\\command}
\TheProperty*[mail!postfix!mysql!restart\_command]{restart\_command \Arg{command}}

The \Arg{command} to restart the service. Can be a full path or
just the command name that can be found in the current search path. 
For the distribution
\begin{inparaitem}
\item[\TheDistribution{ubuntu}] Ubuntu 10.04
\end{inparaitem}
the value is set to \qcode{/etc/init.d/postfix restart}.

%% postalias_command
\TheProperty[mail:postfix:mysql:postalias_command]{postalias\_\\command}
\TheProperty*[mail!postfix!mysql!postalias\_command]{postalias\_command \Arg{command}}

The \Arg{command} for the \code{postalias} command. Can be a full path or
just the command name that can be found in the current search path.
For the distribution
\begin{inparaitem}
\item[\TheDistribution{ubuntu}] Ubuntu 10.04
\end{inparaitem}
the value is set to \qcode{/usr/sbin/postalias}.

%% postmap_command
\TheProperty[mail:postfix:mysql:postmap_command]{postmap\_\\command}
\TheProperty*[mail!postfix!mysql!postmap\_command]{postmap\_command \Arg{command}}

The \Arg{command} for the \code{postmap} command. Can be a full path or
just the command name that can be found in the current search path.
For the distribution
\begin{inparaitem}
\item[\TheDistribution{ubuntu}] Ubuntu 10.04
\end{inparaitem}
the value is set to \qcode{/usr/sbin/postmap}.

%% mysql_command
\TheProperty[mail:postfix:mysql:mysql_command]{mysql\_command}
\TheProperty*[mail!postfix!mysql!mysql\_command]{mysql\_command \Arg{command}}

The \Arg{command} for the \code{mysql} command. Can be a full path or
just the command name that can be found in the current search path.
For the distribution
\begin{inparaitem}
\item[\TheDistribution{ubuntu}] Ubuntu 10.04
\end{inparaitem}
the value is set to \qcode{/usr/bin/mysql}.

%% chown_command
\TheProperty[mail:postfix:mysql:chown_command]{chown\_command}
\TheProperty*[mail!postfix!mysql!chown\_command]{chown\_command \Arg{command}}

The \Arg{command} for the \code{chown} command. Can be a full path or
just the command name that can be found in the current search path.
For the distribution
\begin{inparaitem}
\item[\TheDistribution{ubuntu}] Ubuntu 10.04
\end{inparaitem}
the value is set to \qcode{/bin/chown}.

%% mailname_file
\TheProperty[mail:postfix:mysql:mailname_file]{mailname\_\\file}
\TheProperty*[mail!postfix!mysql!mailname\_file]{mailname\_file \Arg{path}}

The \Arg{path} of the \code{mailname} file. Should be the full path 
of the file.
For the distribution
\begin{inparaitem}
\item[\TheDistribution{ubuntu}] Ubuntu 10.04
\end{inparaitem}
the value is set to \qcode{/etc/mailname}.

%% main_file
\TheProperty[mail:postfix:mysql:main_file]{main\_file}
\TheProperty*[mail!postfix!mysql!main\_file]{main\_file \Arg{path}}

The \Arg{path} of the main configuration file of postfix.
If the path is not absolute then the path is assumed to be
in the configuration directory.
Default is the file \qcode{main.cf} in the configuration directory.

%% master_file
\TheProperty[mail:postfix:mysql:master_file]{master\_file}
\TheProperty*[mail!postfix!mysql!master\_file]{master\_file \Arg{path}}

The \Arg{path} of the the master file of postfix configures
the postfix master process. 
If the path is not absolute then the path is assumed to be
in the configuration directory.
Default is the file \qcode{master.cf} in the configuration directory.

%% configuration_directory
\TheProperty[mail:postfix:mysql:configuration_directory]{configuration\_\\directory}
\TheProperty*[mail!postfix!mysql!configuration\_directory]{configuration\_directory \Arg{path}}

The \Arg{path} of the configuration directory of the postfix storage. 
For the distribution
\begin{inparaitem}
\item[\TheDistribution{ubuntu}] Ubuntu 10.04
\end{inparaitem}
the value is set to \qcode{/etc/postfix}.

%% mailbox_base_directory
\TheProperty[mail:postfix:mysql:mailbox_base_directory]{mailbox\_base\_\\directory}
\TheProperty*[mail!postfix!mysql!mailbox\_base\_directory]{mailbox\_base\_directory \Arg{path}}

The \Arg{path} of the base directory for virtual mailbox.
In the directory the mailbox is created for each virtual domain and user.
Should be an absolute path.
For the distribution
\begin{inparaitem}
\item[\TheDistribution{ubuntu}] Ubuntu 10.04
\end{inparaitem}
the value is set to \qcode{/var/mail/vhosts}.

%% packages
\TheProperty[mail:postfix:mysql:packages]{packages}
\TheProperty*[mail!postfix!mysql!packages]{packages \Arg{packages}}

The \Arg{packages} list that needed to be installed for the MySQL storage.
For the distribution
\begin{inparaitem}
\item[\TheDistribution{ubuntu}] Ubuntu 10.04
\end{inparaitem}
the value is set to \qcode{postfix, postfix-mysql, mysql-client, mysql-server}.

%% alias_maps
\TheProperty[mail:postfix:mysql:alias_maps]{alias\_maps}
\TheProperty*[mail!postfix!mysql!alias\_maps]{alias\_maps \Arg{uri}}

The \Arg{uri} that points to the alias maps hash file.
For the distribution
\begin{inparaitem}
\item[\TheDistribution{ubuntu}] Ubuntu 10.04
\end{inparaitem}
the value is set to \qcode{hash://etc/postfix/aliases}.

%% alias_database
\TheProperty[mail:postfix:mysql:alias_database]{alias\_database}
\TheProperty*[mail!postfix!mysql!alias\_database]{alias\_database \Arg{uri}}

The \Arg{uri} that points to the alias databases hash file.
For the distribution
\begin{inparaitem}
\item[\TheDistribution{ubuntu}] Ubuntu 10.04
\end{inparaitem}
the value is set to \qcode{hash://etc/postfix/aliases}.

%% virtual_mailbox_maps
\TheProperty[mail:postfix:mysql:virtual_mailbox_maps]{virtual\_\\mailbox\_maps}
\TheProperty*[mail!postfix!mysql!virtual\_mailbox\_maps]{virtual\_mailbox\_maps \Arg{uri}}

The \Arg{uri} that points to the virtual mailbox maps MySQL configuration file.
For the distribution
\begin{inparaitem}
\item[\TheDistribution{ubuntu}] Ubuntu 10.04
\end{inparaitem}
the value is set to \qcode{mysql://etc/postfix/mysql\_mailbox.cf}.

%% virtual_alias_maps
\TheProperty[mail:postfix:mysql:virtual_alias_maps]{virtual\_\\alias\_maps}
\TheProperty*[mail!postfix!mysql!virtual\_alias\_maps]{virtual\_alias\_maps \Arg{uri}}

The \Arg{uri} that points to the virtual alias maps MySQL configuration file.
For the distribution
\begin{inparaitem}
\item[\TheDistribution{ubuntu}] Ubuntu 10.04
\end{inparaitem}
the value is set to \qcode{mysql://etc/postfix/mysql\_alias.cf}.

%% virtual_mailbox_domains
\TheProperty[mail:postfix:mysql:virtual_mailbox_domains]{virtual\_\\mailbox\_domains}
\TheProperty*[mail!postfix!mysql!virtual\_mailbox\_domains]{virtual\_mailbox\_domains \Arg{uri}}

The \Arg{uri} that points to the virtual mailbox domains MySQL configuration file.
For the distribution
\begin{inparaitem}
\item[\TheDistribution{ubuntu}] Ubuntu 10.04
\end{inparaitem}
the value is set to \qcode{mysql://etc/postfix/mysql\_domains.cf}.

%% minimum_uid
\TheProperty[mail:postfix:mysql:minimum_uid]{minimum\_uid}
\TheProperty*[mail!postfix!mysql!minimum\_uid]{minimum\_uid \Arg{id}}

The minimum user \Arg{id} for virtual users.
For the distribution
\begin{inparaitem}
\item[\TheDistribution{ubuntu}] Ubuntu 10.04
\end{inparaitem}
the value is set to \qcode{5000}.

%% virtual_uid
\TheProperty[mail:postfix:mysql:virtual_uid]{virtual\_uid}
\TheProperty*[mail!postfix!mysql!virtual\_uid]{virtual\_uid \Arg{id}}

The user \Arg{id} for virtual users.
For the distribution
\begin{inparaitem}
\item[\TheDistribution{ubuntu}] Ubuntu 10.04
\end{inparaitem}
the value is set to \qcode{5000}.

%% virtual_gid
\TheProperty[mail:postfix:mysql:virtual_gid]{virtual\_gid}
\TheProperty*[mail!postfix!mysql!virtual\_gid]{virtual\_gid \Arg{id}}

The group \Arg{id} for virtual users.
For the distribution
\begin{inparaitem}
\item[\TheDistribution{ubuntu}] Ubuntu 10.04
\end{inparaitem}
the value is set to \qcode{5000}.

%% charset
\TheProperty[mail:postfix:mysql:charset]{charset}
\TheProperty*[mail!postfix!mysql!charset]{charset \Arg{name}}

The \Arg{name} of the character set for the files.
For the distribution
\begin{inparaitem}
\item[\TheDistribution{ubuntu}] Ubuntu 10.04
\end{inparaitem}
the value is set to \qcode{utf-8}.

