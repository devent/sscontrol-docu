
\begin{lstlisting}[style=Java,label=lst:dhclient_example_script,
title={Example script for the Dhclient service. It will remove the domain name servers from the DHCP request and use the DNS server on the localhost.}]
dhclient {
    requests "!domain-name-servers"
    prepend "domain-name-servers", "127.0.0.1"
}
\end{lstlisting}

\begin{lstlisting}[style=Java,label=lst:dhclient_ubuntu_profile_min,
title={Minimal Ubuntu Dhclient profile, all needed profile properties are already set to sensible default values.}]
profile "ubuntu_10_04", {
    dhclient {
    }
}
\end{lstlisting}

\begin{lstlisting}[style=rcfile_nonumbers,
label=lst:dhclient_maincf_example,
title={Example of the DHCP configuration file that is created from the Dhclient profile.
The file is saved as /etc/dhcp3/dhclient.conf}]

# SSCONTROL-dhclient
option rfc3442-classless-static-routes code 121 = array of unsigned integer 8;
# SSCONTROL-dhclient-END


# SSCONTROL-dhclient
send host-name "<hostname>";
# SSCONTROL-dhclient-END

#send dhcp-client-identifier 1:0:a0:24:ab:fb:9c;
#send dhcp-lease-time 3600;
#supersede domain-name "fugue.com home.vix.com";
# SSCONTROL-dhclient
prepend domain-name-servers 127.0.0.1
# SSCONTROL-dhclient-END

# SSCONTROL-dhclient
request subnet-mask, broadcast-address, time-offset, routers, domain-name, domain-search, host-name, netbios-name-servers, netbios-scope, interface-mtu, rfc3442-classless-static-routes, ntp-servers, dhcp6.domain-search, dhcp6.fqdn, dhcp6.name-servers, dhcp6.sntp-servers;
# SSCONTROL-dhclient-END

\end{lstlisting}

