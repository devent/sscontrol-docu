\subsection{Firewall Script}

\paragraph{firewall}

\statement{firewall}{firewall \{...\}}

Entry point in the firewall script.

\paragraph{allow}

\statement*{allow [[port] [, proto]]}\\
\statement*{[from [, port] [, proto]]}

Allow packets from the source host, port and protocol to the destination
host, port and protocol.

\paragraph{allow}

\statement{firewall!allow}{allow}

Without any arguments sets the default mode for the firewall to allow all.

\paragraph{deny}

\statement*{deny [[port] [, proto]]}\\
\statement*{[from [, port] [, proto]]}

Denies packets from the source host, port and protocol to the destination
host, port and protocol.

\paragraph{deny}

\statement{firewall!deny}{deny}

Without any arguments sets the default mode for the firewall to deny all.

\paragraph{port}

\statement{firewall!port}{port \typeint$|$\typestring}

Sets the port as port number or service name to allow or deny packets from.

\paragraph{proto}

\statement{firewall!proto}{proto tcp$|$udp}

Sets the Internet protocol to allow or deny packets from. If no protocol
is set TCP and UDP is assumed.

\begin{compactdesc}
\item[tcp] TCP/IP protocol.
\item[udp] UDP/IP protocol.
\end{compactdesc}

\paragraph{from}

\statement{firewall!from}{from any$|$\typestring{}}

Sets the host as the source. Keyword ``any'' means from any host.

\paragraph{to}

\statement{firewall!to}{to any$|$\typestring{} [, port] [, proto]}

Sets the host as the destination. Keyword ``any'' means from any host.

