\label{sec:dhclient_profile}
\subsection{Dhclient Profile}

%% install_command
\TheProperty[dhclient:install_command]{install\_\\command}
\TheProperty*[dhclient!install\_command]{install\_command \Arg{command}}

The \Arg{command} to install package on the system. Can be a full path or
just the command name that can be found in the current search path. 
For the distributions
\begin{inparaitem}
\item[\TheDistribution{ubuntu}] Ubuntu 12.04\footnote{\TheUbuntuPreciseLTSDate},
\item[\TheDistribution{ubuntu}] Ubuntu 14.04\footnote{\TheUbuntuTrustyLTSDate}
\end{inparaitem}
the value is set to \qcode{/usr/bin/aptitude}.

%% restart_command
\TheProperty[dhclient:restart_command]{restart\_\\command}
\TheProperty*[dhclient!restart\_command]{restart\_command \Arg{command}}

The \Arg{command} to restart the service. Can be a full path or
just the command name that can be found in the current search path. 
The placeholder \qcode{<interfaces>} is replaced with the list of network
interfaces that needs to be restarted.

For the distributions
\begin{inparaitem}
\item[\TheDistribution{ubuntu}] Ubuntu 12.04
the value is set to \qcode{/etc/init.d/networking restart}.
\end{inparaitem}
\begin{inparaitem}
\item[\TheDistribution{ubuntu}] Ubuntu 14.04
the value is set to \qcode{<interfaces:\{inet|ifdown <inet> \&\& ifup <inet>\}>}.
\end{inparaitem}

%% configuration_directory
\TheProperty[dhclient:configuration_directory]{configuration\_\\directory}
\TheProperty*[dhclient!configuration\_directory]{configuration\_directory \Arg{path}}

The \Arg{path} of the configuration directory of the Dhclient service. 
For the distributions
\begin{inparaitem}
\item[\TheDistribution{ubuntu}] Ubuntu 12.04,
\item[\TheDistribution{ubuntu}] Ubuntu 14.04
\end{inparaitem}
the value is set to \qcode{/etc/dhcp}.

%% configuration_file
\TheProperty[dhclient:configuration_file]{configuration\_\\file}
\TheProperty*[dhclient!configuration\_file]{configuration\_file \Arg{name}}

The \Arg{name} of the configuration file of the Dhclient service. 
The file is located in the configuration directory.
For the distribution
\begin{inparaitem}
\item[\TheDistribution{ubuntu}] Ubuntu 12.04,
\item[\TheDistribution{ubuntu}] Ubuntu 14.04
\end{inparaitem}
the value is set to \qcode{dhclient.conf}.

%% system_name
\TheProperty[dhclient:mysql:system_name]{system\_name}
\TheProperty*[dhclient!mysql!system\_name]{system\_name \Arg{name}}

The system \Arg{name}. 
For the distribution
\begin{inparaitem}
\item[\TheDistribution{ubuntu}] Ubuntu 12.04,
\item[\TheDistribution{ubuntu}] Ubuntu 14.04
\end{inparaitem}
the value is set to \qcode{ubuntu}.

%% packages
\TheProperty[dhclient:packages]{packages}
\TheProperty*[dhclient!packages]{packages \Arg{packages}}

The \Arg{packages} list that needed to be installed for the Dhclient service.
For the distribution
\begin{inparaitem}
\item[\TheDistribution{ubuntu}] Ubuntu 12.04,
\item[\TheDistribution{ubuntu}] Ubuntu 14.04
\end{inparaitem}
the value is set to \qcode{dhcp3-client}.

%% restart_services
\TheProperty[dhclient:restart_services]{restart\_\\services}
\TheProperty*[dhclient!restart\_services]{restart\_services \Arg{services}}

The list of \Arg{services} to restart the dhclient.
For the distribution
\begin{inparaitem}
\item[\TheDistribution{ubuntu}] Ubuntu 12.04,
\item[\TheDistribution{ubuntu}] Ubuntu 14.04,
\end{inparaitem}
the value is set to \qcode{}, empty, i.e. no services.

%% restart_interfaces
\TheProperty[dhclient:restart_interfaces]{restart\_\\interfaces}
\TheProperty*[dhclient!restart\_interfaces]{restart\_interfaces \Arg{interfaces}}

The list of \Arg{interfaces} to restart.
For the distribution
\begin{inparaitem}
\item[\TheDistribution{ubuntu}] Ubuntu 14.04
\end{inparaitem}
the value is set to \qcode{eth0}.

%% charset
\TheProperty[dhclient:charset]{charset}
\TheProperty*[dhclient!charset]{charset \Arg{name}}

Sets the default character set \Arg{name} in which configuration files are 
saved. For the distributions
\begin{inparaitem}
\item[\TheDistribution{ubuntu}] Ubuntu 12.04,
\item[\TheDistribution{ubuntu}] Ubuntu 14.04
\end{inparaitem}
the value is set to \qcode{utf-8}.

%% default_option
\TheProperty[dhclient:default_option]{default\_option}
\TheProperty*[dhclient!default\_option]{default\_option \Arg{option}}

Sets the default \Arg{option} for the Dhclient. For the distributions
\begin{inparaitem}
\item[\TheDistribution{ubuntu}] Ubuntu 12.04,
\item[\TheDistribution{ubuntu}] Ubuntu 14.04
\end{inparaitem}
the value is set to \qcode{rfc3442-classless-static-routes code 121 = array of unsigned integer 8}.

%% default_sends
\TheProperty[dhclient:default_sends]{default\_sends}
\TheProperty*[dhclient!default\_sends]{default\_sends \Arg{sends}}

Sets the default \Arg{sends} statement for the Dhclient. For the distributions
\begin{inparaitem}
\item[\TheDistribution{ubuntu}] Ubuntu 12.04,
\item[\TheDistribution{ubuntu}] Ubuntu 14.04
\end{inparaitem}
the value is set to \qcode{host-name "<hostname>"}.

%% default_requests
\TheProperty[dhclient:default_requests]{default\_requests}
\TheProperty*[dhclient!default\_requests]{default\_requests \Arg{requests}}

Sets the default \Arg{requests} statement for the Dhclient. For the distributions
\begin{inparaitem}
\item[\TheDistribution{ubuntu}] Ubuntu 12.04,
\item[\TheDistribution{ubuntu}] Ubuntu 14.04
\end{inparaitem}
the value is set to \qcode{subnet-mask, broadcast-address, time-offset, routers,\
domain-name, domain-name-servers, domain-search, host-name,\
dhcp6.name-servers, dhcp6.domain-search,\
netbios-name-servers, netbios-scope, interface-mtu,\
rfc3442-classless-static-routes, ntp-servers,\
dhcp6.fqdn, dhcp6.sntp-servers}.

