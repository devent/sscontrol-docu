\subsection{Hosts Service}

Unter Linux Systemen ist es üblich in der \code{/etc/hosts} Datei bekannte
Hostnamen und deren IP-Adressen zu speichern. Mit dieser Datei kann das System
Hostnamen auflösen ohne einen DNS-Server.

Mit dem Hosts Service können wir diese statische Liste der bekannten Hostnamen und
deren IP Adresse konfigurieren lassen.

\subsubsection{Hosts Profil}

Das Hostname Profil wird eingeleitet durch die \directive{hosts}. Gefolgt von
den folgenden zwei Direktiven:

\paragraph{\directive{configuration\_directory}}

beschreibt in welchem Verzeichnis
sich die Konfiguration für die Hosts befinden. Unter Ubuntu und Debian
Systemen wird die Konfiguration in der Datei \code{/etc/hosts} gespeichert,
somit ist es das Verzeichnis \code{/etc}.

\paragraph{\directive{status\_command}}

enthält das Kommando mit dem man die eingestellten Hosts abfragen kann. Es gibt
eigentlich kein Kommando, dass so einen Status abfragen kann da es nur normalerweise
nur die Datei \code{/etc/hosts} gibt. Wir können uns daher dem Kommando \code{cat}
bedienen, dass den Inhalt der Datei \code{/etc/hosts} auslesen kann.

\begin{lstlisting}[style=Java, caption=Beispiel Hosts Profil Ubuntu Server]
profiles {
	create_profile("ubuntu") {
		hostname {
			configuration_directory '/etc'
			status_command '/bin/cat /etc/hosts'
		}
	}
}
\end{lstlisting}

\subsubsection{Hosts Script}

Das Script enthält ein oder mehrere \directive{host}, dass ein oder mehrere
\directive{alias} anthalten kann.

\paragraph{\directive{hosts}}

Repräsentiert einen neuen Host mit der dazugehörigen IP Adresse. Braucht die zwei
Parameter:

\begin{asparadesc}
\item[\code{ip}] eine Zeichenfolge mit der IP Adresse des Host und
\item[\code{hostname}] der Name des Host, als eine Zeichenfolge.
\end{asparadesc}

\paragraph{\directive{alias}}

eine Zeichenfolge, die den Pseudonym des Host angibt.

\subsubsection*{Beispiel Listing}

\begin{lstlisting}[style=Java, caption=Beispiel Hosts Script]
server {
	host ip: '127.0.0.1', hostname: 'localhost'
	host(ip: '192.168.0.49', hostname: 'srv1a.globalscalingsoftware.com') {
		alias 'srv1'
		alias 'srv1a'
	}
	host(ip: '192.168.0.50', hostname: 'srv1b.globalscalingsoftware-projects.com') {
		alias 'srv1b'
	}
}
\end{lstlisting}

\begin{asparadesc}
\item[Zeile 2] zeigt die einfachste Form der \directive{host} an. Es wird ein Hosteintrag
erstellt, mit dem Hostnamen ``localhost'' und der IP Adresse ``127.0.0.1'';
\item[Zeile 3 und 6] zeigt die die \directive{host} mit einem oder mehreren Pseudonymen für einen Host.
\end{asparadesc}

