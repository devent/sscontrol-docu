\subsection{Database Service}

Mit diesem Service kann man einen Datenbankserver, wie MySQL oder PostgreSQL,
einrichten. Es wird ein Datenbankserver installieren und Datenbanken
und Benutzer eingerichtet.

\subsubsection{Database Script}

Das Database Profil wird eingeleitet durch die \directive{database}. Gefolgt von
den folgenden Direktiven:

\paragraph{\directive{bind\_address}}

die IP Adresse, an die der Server gebunden wird. Der Server empfängt nur auf diese
Adresse Anfragen. Die Adresse kann entweder eine IPv4, IPv6 Adresse oder
ein Hostnamen sein.

Die Direktive erwartet die Adresse als den ersten Parameter als eine Zeichenfolge.

\paragraph{\directive{admin\_password}}

der Administratorpasswort des Datenbankservers. Der Administrator ist der Benutzer,
der vollen Zugriff auf den Server hat. Der Database-Service wird beim installieren
des Datenbankservers dieses Passwort setzen und verwenden um Datenbanken und Benutzer
zu erstellen.

Die Direktive erwartet das Passwort als den ersten Parameter als eine Zeichenfolge.

\paragraph{\directive{database}}

bezeichnet eine neue Datenbank. Die Datenbank wird auf dem Server erstellt, falls
sie noch nicht existiert.

Die Direktive erwartet den Namen der Datenbank als den ersten Parameter als eine Zeichenfolge.

\paragraph{\directive{user}}

bezeichnet einen neuen Benutzer. Der Benutzer wird auf dem Server erstellt, falls er
noch nicht existiert. Die Direktive kann ein oder mehrere \directive{use\_database}
enthalten, die bezeichnen für welche Datenbank der Benutzer Zugriff haben soll.

Die Direktive erwartet zwei Parameter:

\begin{asparadesc}
\item[\code{name}] der Name des Benutzers, als eine Zeichenfolge und
\item[\code{password}] der Password des neuen Benutzers, als eine Zeichenfolge.
\end{asparadesc}

Ein weitere optionales Parameter kann angegeben werden:

\begin{asparadesc}
\item[\code{server}] der Servername des Benutzers, als eine Zeichenfolge. Standardeinstellung
ist ``localhost''.
\end{asparadesc}

\paragraph{\directive{use\_database}}

bezeichnet die Datenbank, die auf die der Benutzer vollen Zugriff haben soll. Dem Benutzer
werden alle üblichen Rechte zugewiesen. Die Datenbank muss entweder mit der \directive{database}
erstellt werden oder muss bereits existieren.

Erwartet den Namen der Datenbank als eine Zeichenfolge im ersten Parameter.

\subsubsection*{Beispiel Listing}

\begin{lstlisting}[style=Java, caption=Beispiel Database Script]
server {
    database {
        bind_address "127.0.0.1"
        admin_password "mysqladminpassword"
        database "drupal6db"
        user "test1" password "test1password"
        user "test2" password "test2password" server "srv1"
        user "drupal6" password "drupal6password" server "localhost", {
            use_database "drupal6db"
        }
        user "wordpress" password "wordpresspassword", {
            use_database "wordpresdb"
        }
    }
}
\end{lstlisting}

\begin{asparadesc}
\item[Zeile 3] Bindet den Datenbankserver an die Adresse ``127.0.0.1'';
\item[Zeile 4] Setzt das Administratorpasswort des Servers;
\item[Zeile 5] Erstellt eine neue Datenbank;
\item[Zeile 6] Erstellt einen neuen Benutzer an ``localhost'' (implizit);
\item[Zeile 7] Erstellt einen neuen Benutzer an ``srv1'';
\item[Zeile 8] Erstellt einen neuen Benutzer an ``localhost'' (explizit) und
erlaubt alle üblichen Rechte an der Datenbank ``drupal6db'';
\item[Zeile 11] Erstellt einen neuen Benutzer an ``localhost'' (implizit) und
erlaubt alle üblichen Rechte an der Datenbank ``wordpresdb'', wobei die Datenbank
bereits zuvor erstellt worden ist;
\end{asparadesc}

