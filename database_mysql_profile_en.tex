\label{sec:mysql_profile}
\subsection{MySQL Profile}

%% install_command
\TheProperty[database:mysql:install_command]{install\_\\command}
\TheProperty*[database!mysql!install\_command]{install\_command \Arg{command}}

The \Arg{command} to install package on the system. Can be a full path or
just the command name that can be found in the current search path. 
For the distribution
\begin{inparaitem}
\item[\TheDistribution{ubuntu}] Ubuntu 10.04,
\item[\TheDistribution{ubuntu}] Ubuntu 12.04
\end{inparaitem}
the value is set to \qcode{export DEBIAN\_FRONTEND=noninteractive \textbackslash{n} /usr/bin/aptitude -q update \&\& /usr/bin/aptitude -q -y install}.

%% restart_command
\TheProperty[database:mysql:restart_command]{restart\_\\command}
\TheProperty*[database!mysql!restart\_command]{restart\_command \Arg{command}}

The \Arg{command} to restart the service. Can be a full path or
just the command name that can be found in the current search path. 
For the distribution
\begin{inparaitem}
\item[\TheDistribution{ubuntu}] Ubuntu 10.04,
\item[\TheDistribution{ubuntu}] Ubuntu 12.04
\end{inparaitem}
the value is set to \qcode{/sbin/restart}.

%% mysqladmin_command
\TheProperty[database:mysql:mysqladmin_command]{mysqladmin\_\\command}
\TheProperty*[database!mysql!mysqladmin\_command]{mysqladmin\_command \Arg{name}}

The \Arg{command} to \code{mysqladmin} command that is used to administer the 
MySQL database. Can be a full path or
just the command name that can be found in the current search path. 
For the distribution
\begin{inparaitem}
\item[\TheDistribution{ubuntu}] Ubuntu 10.04,
\item[\TheDistribution{ubuntu}] Ubuntu 12.04
\end{inparaitem}
the value is set to \qcode{/usr/bin/mysqladmin}.

%% mysql_command
\TheProperty[database:mysql:mysql_command]{mysql\_command}
\TheProperty*[database!mysql!mysql\_command]{mysql\_command \Arg{name}}

The \Arg{command} to \code{mysql} command that is used to create databases,
create users and set privileges. Can be a full path or
just the command name that can be found in the current search path. 
For the distribution
\begin{inparaitem}
\item[\TheDistribution{ubuntu}] Ubuntu 10.04,
\item[\TheDistribution{ubuntu}] Ubuntu 12.04
\end{inparaitem}
the value is set to \qcode{/usr/bin/mysql}.

%% configuration_directory
\TheProperty[database:mysql:configuration_directory]{configuration\_\\directory}
\TheProperty*[database!mysql!configuration\_directory]{configuration\_directory \Arg{path}}

The \Arg{path} of the MySQL configuration directory.
For the distribution
\begin{inparaitem}
\item[\TheDistribution{ubuntu}] Ubuntu 10.04,
\item[\TheDistribution{ubuntu}] Ubuntu 12.04
\end{inparaitem}
the value is set to \qcode{/etc/mysql/conf.d}.

%% mysqld_configuration_file
\TheProperty[database:mysql:mysqld_configuration_file]{mysqld\_\\configuration\_\\file}
\TheProperty*[database!mysql!mysqld\_configuration\_file]{mysqld\_configuration\_file \Arg{name}}

The \Arg{name} of the custom MySQL configuration file. The file is saved
under the configuration directory.
For the distribution
\begin{inparaitem}
\item[\TheDistribution{ubuntu}] Ubuntu 12.04,
\item[\TheDistribution{ubuntu}] Ubuntu 10.04
\end{inparaitem}
the value is set to \qcode{sscontrol\_mysqld.cnf}.

%% packages
\TheProperty[database:mysql:packages]{packages}
\TheProperty*[database!mysql!packages]{packages \Arg{packages}}

The list of \Arg{packages} that are needed for the MySQL database. 
For the distribution
\begin{inparaitem}
\item[\TheDistribution{ubuntu}] Ubuntu 12.04,
\item[\TheDistribution{ubuntu}] Ubuntu 10.04
\end{inparaitem}
the value is set to \qcode{mysql-server, mysql-client}.

%% restart_services
\TheProperty[database:mysql:restart_services]{restart\_\\services}
\TheProperty*[database!mysql!restart\_services]{restart\_services \Arg{services}}

The list of \Arg{services} to restart the MySQL database.
For the distribution
\begin{inparaitem}
\item[\TheDistribution{ubuntu}] Ubuntu 12.04,
\item[\TheDistribution{ubuntu}] Ubuntu 10.04
\end{inparaitem}
the value is set to \qcode{mysql}.

%% charset
\TheProperty[database:mysql:charset]{charset}
\TheProperty*[database!mysql!charset]{charset \Arg{name}}

Sets the default character set \Arg{name} in which configuration files are 
saved. For the distribution
\begin{inparaitem}
\item[\TheDistribution{ubuntu}] Ubuntu 10.04,
\item[\TheDistribution{ubuntu}] Ubuntu 12.04
\end{inparaitem}
the value is set to \qcode{utf-8}.

%% default_character_set
\TheProperty[database:mysql:default_character_set]{default\_\\character\_\\set}
\TheProperty*[database!mysql!default\_character\_set]{default\_character\_set \Arg{name}}

The \Arg{name} of the character set to use per default for new created databases.
For the distribution
\begin{inparaitem}
\item[\TheDistribution{ubuntu}] Ubuntu 12.04,
\item[\TheDistribution{ubuntu}] Ubuntu 10.04
\end{inparaitem}
the value is set to \qcode{utf8}.

%% default_collate
\TheProperty[database:mysql:default_collate]{default\_collate}
\TheProperty*[database!mysql!default\_collate]{default\_collate \Arg{name}}

The \Arg{name} of the collate to use per default for new created databases.
For the distribution
\begin{inparaitem}
\item[\TheDistribution{ubuntu}] Ubuntu 12.04,
\item[\TheDistribution{ubuntu}] Ubuntu 10.04
\end{inparaitem}
the value is set to \qcode{utf8\_general\_ci}.
 
%% default_debug
\TheProperty[database:mysql:default_debug]{default\_debug}
\TheProperty*[database!mysql!default\_debug]{default\_debug \Arg{debug}}

The default \Arg{debug} logging.
For the distribution
\begin{inparaitem}
\item[\TheDistribution{ubuntu}] Ubuntu 10.04,
\item[\TheDistribution{ubuntu}] Ubuntu 12.04
\end{inparaitem}
the value is set to \qcode{level: 0}.

%% default_bind_address
\TheProperty[database:mysql:default_bind_address]{default\_bind\_\\address}
\TheProperty*[database!mysql!default\_bind\_address]{default\_bind\_address \Arg{address}}

The default IP \Arg{address} or host name on which the database server should
listen for connections. For the distribution
\begin{inparaitem}
\item[\TheDistribution{ubuntu}] Ubuntu 10.04,
\item[\TheDistribution{ubuntu}] Ubuntu 12.04
\end{inparaitem}
the value is set to \qcode{127.0.0.1}.
 
%% default_user_server
\TheProperty[database:mysql:default_user_server]{default\_user\_\\server}
\TheProperty*[database!mysql!default\_user\_server]{default\_user\_server \Arg{server}}

The default \Arg{server} that is used for database users.
For the distribution
\begin{inparaitem}
\item[\TheDistribution{ubuntu}] Ubuntu 12.04,
\item[\TheDistribution{ubuntu}] Ubuntu 10.04
\end{inparaitem}
the value is set to \qcode{localhost}.
 
