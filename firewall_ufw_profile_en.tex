\label{sec:firewall_profile}
\subsection{UFW Profile}

%% install_command
\TheProperty[firewall:ufw:install_command]{install\_\\command}
\TheProperty*[firewall!ufw!install\_command]{install\_command \Arg{command}}

The \Arg{command} to install package on the system. Can be a full path or
just the command name that can be found in the current search path. 
For the distribution
\begin{inparaitem}
\item[\TheDistribution{ubuntu}] Ubuntu 10.04,
\item[\TheDistribution{ubuntu}] Ubuntu 12.04
\end{inparaitem}
the value is set to \qcode{export DEBIAN\_FRONTEND=noninteractive\textbackslash{n} /usr/bin/aptitude -q update \&\& /usr/bin/aptitude -q -y install}.

%% ufw_command
\TheProperty[firewall:ufw:ufw_command]{ufw\_command}
\TheProperty*[firewall!ufw!ufw\_command]{ufw\_command \Arg{command}}

The \Arg{command} that is used to setup the firewall rules. Can be a full path or
just the command name that can be found in the current search path. 
For the distribution
\begin{inparaitem}
\item[\TheDistribution{ubuntu}] Ubuntu 10.04,
\item[\TheDistribution{ubuntu}] Ubuntu 12.04
\end{inparaitem}
the value is set to \qcode{/usr/sbin/ufw}.

%% packages
\TheProperty[firewall:ufw:packages]{packages}
\TheProperty*[firewall!ufw!packages]{packages \Arg{packages}}

The list of \Arg{packages} that are needed for the UFW firewall. 
For the distribution
\begin{inparaitem}
\item[\TheDistribution{ubuntu}] Ubuntu 10.04,
\item[\TheDistribution{ubuntu}] Ubuntu 12.04
\end{inparaitem}
the value is set to \qcode{ufw}.

%% charset
\TheProperty[firewall:ufw:charset]{charset}
\TheProperty*[firewall!ufw!charset]{charset \Arg{name}}

Sets the default character set \Arg{name} in which configuration files are 
saved. For the distribution
\begin{inparaitem}
\item[\TheDistribution{ubuntu}] Ubuntu 10.04,
\item[\TheDistribution{ubuntu}] Ubuntu 12.04
\end{inparaitem}
the value is set to \qcode{utf-8}.
