\begin{lstlisting}[
style=Java,
label=lst:firewall_example_script,
title={Allow firewall rules.}]
firewall {

    // allow from all ports, all addresses to all ports to all addresses
    allow

    // allow port 22/tcp/udp on the host to anywhere
    allow port: 22

    // allow port 22/tcp on the host to anywhere
    allow port: 22, proto: tcp

    // allow port 22/udp on the host to anywhere
    allow port: 22, proto: udp

    // allow port 22/tcp/udp on the host to anywhere
    allow port: "ssh"

    // allow from anywhere to anywhere on the host
    allow from: any to any

    // allow from anywhere port 22 to anywhere port 23 on the host
    allow from: any, port: 22 to any, port: 23

    // allow from anywhere port 22 to anywhere port 23 on the host
    allow from: any, port: "ssh" to any, port: "www"

    // allow from 192.168.0.1 to anywhere port 23 on the host
    allow from: "192.168.0.1" to any, port: 23

    // allow from 192.168.0.1 port 22 to 127.0.0.1 port 23 on the host
    allow from: "192.168.0.1", port: 22 to "127.0.0.1", port: 23

    // allow from 192.168.0.1 port 22/tcp to 127.0.0.1 port 23/udp on the host
    allow from: "192.168.0.1", port: 22, proto: tcp to "127.0.0.1", port: 23, proto: udp

    // allow from 192.168.0.1 port 22/tcp to 127.0.0.1 port 23/udp on the host
    allow from: "192.168.0.1", port: "ssh", proto: tcp to "127.0.0.1", port: 23, proto: udp

    // allow from 192.168.0.1 port ssh/tcp to 127.0.0.1 port http/udp on the host
    allow from: "192.168.0.1", port: "ssh", proto: tcp to "127.0.0.1", port: "http", proto: udp

    // allow from 192.168.0.1 port 22/udp to 127.0.0.1 port 23/tcp on the host
    allow from: "192.168.0.1", port: 22, proto: udp to "127.0.0.1", port: 23, proto: tcp

}
\end{lstlisting}

\begin{lstlisting}[
style=Java,
label=lst:firewall_ubuntu_profile_min,
title={Minimal Ubuntu firewall profile.}]
profile "ubuntu_10_04", {
    firewall {
        service "ufw"
    }
}
\end{lstlisting}
