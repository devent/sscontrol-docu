\subsection{Mail Description}

The mail service conists of multiple processes. The transport or MTA
will receive emails for users and send them to different domains.
There are different transport services available, like
sendmail\cite{sendmail}, postfix\cite{postfix} or exim\cite{exim}.
Only one MTA can operate on the same host and port. The default port
is either the SMTP port 25 or the SMTPS port 465.

Virtual mail accounts can be managed with various backed, like MySQL or
PostgeSQL. If a virtual mail account is used with a database backed then
the database \refsec{sec:database} service can be configured alongside the mail service.

The delivery or MDA is used by the users to get the received emails and read
them on a email client. The MDA is operating on the POP port 110, POPS 995, IMAP
port 143 or IMAPS 993. It is recommended to deprecate the POP for IMAP.
For the delivery there are different services available, like
dovecot\cite{dovecot} or courier\cite{courier}.

The received and sent emails can be checked for spam or viruses.
The services amavisd\cite{amavisd} can unify spam detection and anti-virus.
The services SpamAssassin\cite{SpamAssassin} and postgrey\cite{postgrey}
can be used for spam detection, for example. 
The service clam-av\cite{clam-av} can be used
for virus detection.
