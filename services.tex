\section{Services}

\begin{multicols}{2}

Die Services werden durch Scripte eingerichtet. In den Scripten steht eine
Beschreibung dessen, was man eingerichtet haben möchte, nicht wie. 

Als Beispiel,
ein Httpd Servicescript enthält Anweisungen verschiedene Domains einzurichten,
die verschiedene Seiten liefern. Wir beschreiben nicht wie man den Apache2
Webserver einrichtet, sondern nur die Anweisungen, dass wir diese Domains gerne
hätten. Es ist die Aufgabe des \sscontrol{} die Konfiguration für den Apach2
Webserver zu erstellen und ihn einzurichten. Diese Aufgabe löst \sscontrol{} mit
Hilfe der Profile.

Dadurch dass wir diese Konfiguration abstrahieren können wir das benutzte Profil
austauschen. Z.B. Profil A enthält Anweisungen wie man einen Ubuntu Server mit
dem Apache2 Webserver einrichtet und Profil B enthält Anweisungen wie man einen
RedHat Server mit Lighthttpd einrichtet. Man kann nun Profil A durch Profil B
ersetzen und \sscontrol{} wird die gleichen Domains mit den gleichen Seiten
einrichten.

\end{multicols}

\subsection{Hostname Service}

Der Hostname wird von verschiedenen Services verwenden um den Namen der Maschine
zu erhalten. Mit diesem Service kann man diesen Namen setzen lassen.

\subsubsection{Hostname Profil}

Das Hostname Profil wird eingeleitet durch die \directive{hostname}. Gefolgt von
den folgenden drei Direktiven:

\paragraph{\directive{configuration\_directory}}

beschreibt in welchem Verzeichnis
sich die Konfiguration für den Hostnamen befinden. Unter Ubuntu und Debian
Systemen wird die Konfiguration in der Datei \code{/etc/hostname} gespeichert,
somit ist das Verzeichnis \code{/etc}.

\paragraph{\directive{restart\_command}}

enthält das Kommando mit dem man im
laufendem Betrieb den Hostnamen ändern kann.  Unter Ubuntu und Debian
Systemen ist es das Kommando 
\code{/etc/init.d/hostname restart}.

\paragraph{\directive{status\_command}}

enthält das Kommando mit dem man den
aktuellen Hostnamen abfragen kann. Unter Ubuntu und Debian Systemen ist es das
Kommando \code{/bin/hostname}.

\begin{lstlisting}[style=Java, caption=Beispiel Hostname Profil Ubuntu Server]
profiles {
    create_profile("ubuntu") {
        hostname {
            configuration_directory '/etc'
            restart_command '/etc/init.d/hostname restart'
            status_command '/bin/hostname'
        }
    }
}
\end{lstlisting}

\subsubsection{Hostname Script}

Das Script enthält nur eine Direktive, der Name des Hosts.

\paragraph{\directive{hostname}}

erwartet eine einzige Zeichenfolge, der Name des Hosts.

\begin{lstlisting}[style=Java, caption=Beispiel Hostname Script]
server {
	hostname 'srv1'
}
\end{lstlisting}



