\subsection{Hostname Service}

Der Hostname wird von verschiedenen Services verwenden um den Namen der Maschine
zu erhalten. Mit diesem Service kann man diesen Namen setzen lassen.

\subsubsection{Hostname Profil}

Das Hostname Profil wird eingeleitet durch die \directive{hostname}. Gefolgt von
den folgenden drei Direktiven:

\paragraph{\directive{configuration\_directory}}

beschreibt in welchem Verzeichnis
sich die Konfiguration für den Hostnamen befinden. Unter Ubuntu und Debian
Systemen wird die Konfiguration in der Datei \code{/etc/hostname} gespeichert,
somit ist es das Verzeichnis \code{/etc}.

\paragraph{\directive{restart\_command}}

enthält das Kommando mit dem man im
laufendem Betrieb den Hostnamen ändern kann.  Unter Ubuntu und Debian
Systemen ist es das Kommando 
\code{/etc/init.d/hostname restart}.

\paragraph{\directive{status\_command}}

enthält das Kommando mit dem man den
aktuellen Hostnamen abfragen kann. Unter Ubuntu und Debian Systemen ist es das
Kommando \code{/bin/hostname}.

\begin{lstlisting}[style=Java, caption=Beispiel Hostname Profil Ubuntu Server]
profiles {
	create_profile("ubuntu") {
		hostname {
			configuration_directory '/etc'
			restart_command '/etc/init.d/hostname restart'
			status_command '/bin/hostname'
		}
	}
}
\end{lstlisting}

\subsubsection{Hostname Script}

Das Script enthält nur eine Direktive, der Name des Hosts.

\paragraph{\directive{hostname}}

erwartet eine einzige Zeichenfolge, der Name des Hosts.

\begin{lstlisting}[style=Java, caption=Beispiel Hostname Script]
server {
	hostname 'srv1'
}
\end{lstlisting}

