\section{Httpd}

\subsection{Apache 2 Konfigurationsdateien}

Die Apache 2 Konfiguration ist gespeichert in XML Dateien. Es gibt zwei
Gültigkeitsbereich: Die globale Konfiguration und lokale Konfiguration für die
virtuellen Hosts.

SSC sollte die globale Konfiguration verwalten und die virtuellen Hosts. In den
virtuellen Hosts werden die Websiten zu finden sein.

\subsection{Groovy Httpd Scripts}

\subsubsection{Globale Gitweb Konfiguration}

Ein Beispiel des Gitweb Anwendung, konfiguriert im globalem Gültigkeitsbereich.
\code{globalBuilder} ist eine injezierte Variable die ein Groovy-Builder
für den globalen Gültigkeitsbereich representiert.

\begin{multicols}{2}
\begin{lstlisting}[style=Java, caption=Gitweb Httpd Script]
globalBuilder.AddAlias("/git", "/var/www/git")

globalBuilder.AddDirectory("/var/www/git") {
    Options { AddExecCGI }
    Handler("cgi-script", ".cgi")
    DirectoryIndex("gitweb.cgi")
}
\end{lstlisting}

\begin{lstlisting}[style=XML, caption=Gitweb Configuration]
Alias /git /var/www/git

<Directory /var/www/git>
  Options +ExecCGI
  AddHandler cgi-script .cgi
  DirectoryIndex gitweb.cgi
</Directory>
\end{lstlisting}
\end{multicols}

\code{globalBuilder.AddAlias} erstellt eine neuene Alias-Directive.

\code{globalBuilder.AddDirectory} erstellt eine neue Directory-Directive.

\code{Options} nimmt die Optionen wie \config{+ExecCGI} oder \config{-ExecCGI}
auf. Mit \code{AddExecCGI} fügen wir die Option \config{ExecCGI} hinzu. 

\newpage
\subsubsection{Virtuelle Host Konfiguration}

\begin{multicols}{2}
\begin{lstlisting}[style=Java, caption=Virtuel Host Httpd Script]
globalBuilder.NameVirtualHost("*:80")

globalBuilder.VirtualHost("*:80") {
    ServerName("www.domain.tld")
    ServerAlias("domain.tld", "*.domain.tld")
    DocumentRoot("/www/domain")
}

globalBuilder.VirtualHost("*:80") {
    ServerName("www.otherdomain.tld")
    DocumentRoot("/www/otherdomain")
}
\end{lstlisting}

\begin{lstlisting}[style=XML, caption=Virtual Host Configuration]
NameVirtualHost *:80

<VirtualHost *:80>
    ServerName www.domain.tld
    ServerAlias domain.tld *.domain.tld
    DocumentRoot /www/domain
</VirtualHost>

<VirtualHost *:80>
    ServerName www.otherdomain.tld
    DocumentRoot /www/otherdomain
</VirtualHost>
\end{lstlisting}
\end{multicols}

\code{globalBuilder.VirtualHost("*:80")} wird einen neuen virtuellen Host
einleiten.

\code{ServerAlias} nimmt einen Array von Strings entgegen, da es mehrere Aliase
sein können.

