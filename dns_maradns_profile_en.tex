\label{sec:maradns_profile}
\subsection{MaraDNS Profile}

%% install_command
\TheProperty[dns:maradns:install_command]{install\_\\command}
\TheProperty*[dns!maradns!install\_command]{install\_command \Arg{command}}

The \Arg{command} to install package on the system. Can be a full path or
just the command name that can be found in the current search path. 
For the distribution
\begin{inparaitem}
\item[\TheDistribution{ubuntu}] Ubuntu 10.04,
\item[\TheDistribution{ubuntu}] Ubuntu 12.04,
\item[\TheDistribution{ubuntu}] Ubuntu 14.04
\end{inparaitem}
the value is set to \qcode{export DEBIAN\_FRONTEND=noninteractive \textbackslash{n} /usr/bin/aptitude -q update \&\& /usr/bin/aptitude -q -y install}.

%% restart_command
\TheProperty[dns:maradns:restart_command]{restart\_\\command}
\TheProperty*[dns!maradns!restart\_command]{restart\_command \Arg{command}}

The \Arg{command} to restart the service. Can be a full path or
just the command name that can be found in the current search path. 
For the distribution
\begin{inparaitem}
\item[\TheDistribution{ubuntu}] Ubuntu 10.04,
\item[\TheDistribution{ubuntu}] Ubuntu 12.04,
\item[\TheDistribution{ubuntu}] Ubuntu 14.04
\end{inparaitem}
the value is set to \qcode{/etc/init.d/maradns restart}.

%% packages_sources_file
\TheProperty[dns:maradns:packages_sources_file]{packages\_\\sources\_\\file}
\TheProperty*[dns!maradns!packages\_sources\_file]{packages\_sources\_file \Arg{path}}

The \Arg{path} to the packaging configuration directory, that is the directory
where to find the configuration of the repositories.
For the distribution
\begin{inparaitem}
\item[\TheDistribution{ubuntu}] Ubuntu 10.04,
\item[\TheDistribution{ubuntu}] Ubuntu 12.04,
\item[\TheDistribution{ubuntu}] Ubuntu 14.04
\end{inparaitem}
the value is set to \qcode{/etc/apt/sources.list}.

%% configuration_directory
\TheProperty[dns:maradns:configuration_directory]{configuration\_\\directory}
\TheProperty*[dns!maradns!maradns!configuration\_directory]{configuration\_directory \Arg{path}}

The \Arg{path} of the MaraDNS configuration directory.
For the distribution
\begin{inparaitem}
\item[\TheDistribution{ubuntu}] Ubuntu 10.04,
\item[\TheDistribution{ubuntu}] Ubuntu 12.04,
\item[\TheDistribution{ubuntu}] Ubuntu 14.04
\end{inparaitem}
the value is set to \qcode{/etc/maradns}.

%% configuration_file
\TheProperty[dns:maradns:configuration_file]{configuration\_\\file}
\TheProperty*[dns!maradns!maradns!configuration\_file]{configuration\_file \Arg{name}}

The \Arg{name} of the MaraDNS configuration file. The file is saved
under the configuration directory.
For the distribution
\begin{inparaitem}
\item[\TheDistribution{ubuntu}] Ubuntu 10.04,
\item[\TheDistribution{ubuntu}] Ubuntu 12.04,
\item[\TheDistribution{ubuntu}] Ubuntu 14.04
\end{inparaitem}
the value is set to \qcode{mararc}.

%% distribution_name
\TheProperty[dns:maradns:distribution_name]{distribution\_\\name}
\TheProperty*[dns!maradns!distribution\_name]{distribution\_name \Arg{name}}

The \Arg{name} of the distribution, that is used to enable an additional
repository. For the distribution
\begin{inparaitem}
\item[\TheDistribution{ubuntu}] Ubuntu 10.04
\end{inparaitem}
the value is set to \qcode{lucid}.
For the distribution
\begin{inparaitem}
\item[\TheDistribution{ubuntu}] Ubuntu 12.04
\end{inparaitem}
the value is set to \qcode{precise}.
For the distribution
\begin{inparaitem}
\item[\TheDistribution{ubuntu}] Ubuntu 14.04
\end{inparaitem}
the value is set to \qcode{trusty}.

%% additional_repositories
\TheProperty[dns:maradns:additional_repositories]{additional\_\\repositories}
\TheProperty*[dns!maradns!additional\_repositories]{additional\_repositories \Arg{repositories}}

The list of additional \Arg{repositories} that are needed for the service. 
For the distribution
\begin{inparaitem}
\item[\TheDistribution{ubuntu}] Ubuntu 10.04,
\item[\TheDistribution{ubuntu}] Ubuntu 12.04,
\item[\TheDistribution{ubuntu}] Ubuntu 14.04
\end{inparaitem}
the value is set to \qcode{universe}.

%% repository_string
\TheProperty[dns:maradns:repository_string]{repository\_\\string}
\TheProperty*[dns!maradns!repository\_string]{repository\_string \Arg{string}}

The \Arg{string} to enable an additional repository.
For the distribution
\begin{inparaitem}
\item[\TheDistribution{ubuntu}] Ubuntu 10.04,
\item[\TheDistribution{ubuntu}] Ubuntu 12.04,
\item[\TheDistribution{ubuntu}] Ubuntu 14.04
\end{inparaitem}
the value is set to \qcode{deb http://archive.ubuntu.com/ubuntu <distributionName> <repository>}.

%% packages
\TheProperty[dns:maradns:packages]{packages}
\TheProperty*[dns!maradns!packages]{packages \Arg{packages}}

The list of \Arg{packages} that are needed for the MaraDNS.
For the distribution
\begin{inparaitem}
\item[\TheDistribution{ubuntu}] Ubuntu 10.04,
\item[\TheDistribution{ubuntu}] Ubuntu 12.04,
\item[\TheDistribution{ubuntu}] Ubuntu 14.04
\end{inparaitem}
the value is set to \qcode{maradns}.

%% restart_services
\TheProperty[dns:maradns:restart_services]{restart\_\\services}
\TheProperty*[dns!maradns!restart\_services]{restart\_services \Arg{services}}

The list of \Arg{services} to restart MaraDNS.
For the distribution
\begin{inparaitem}
\item[\TheDistribution{ubuntu}] Ubuntu 10.04,
\item[\TheDistribution{ubuntu}] Ubuntu 12.04,
\item[\TheDistribution{ubuntu}] Ubuntu 14.04
\end{inparaitem}
the value is set to \qcode{}, empty, i.e. no services.

%% charset
\TheProperty[dns:maradns:charset]{charset}
\TheProperty*[dns!maradns!charset]{charset \Arg{name}}

Sets the default character set \Arg{name} in which configuration files are 
saved. For the distribution
\begin{inparaitem}
\item[\TheDistribution{ubuntu}] Ubuntu 10.04,
\item[\TheDistribution{ubuntu}] Ubuntu 12.04,
\item[\TheDistribution{ubuntu}] Ubuntu 14.04
\end{inparaitem}
the value is set to \qcode{utf-8}.

%% default_binding_addresses
\TheProperty[dns:maradns:default_binding_addresses]{default\_\\binding\_\\addresses}
\TheProperty*[dns!maradns!default\_binding\_addresses]{default\_binding\_addresses \Arg{addresses}}

Sets the default binding \Arg{addresses}. For the distribution
\begin{inparaitem}
\item[\TheDistribution{ubuntu}] Ubuntu 10.04,
\item[\TheDistribution{ubuntu}] Ubuntu 12.04,
\item[\TheDistribution{ubuntu}] Ubuntu 14.04
\end{inparaitem}
the value is set to \qcode{127.0.0.1}.

%% default_ttl_duration
\TheProperty[dns:maradns:default_ttl_duration]{default\_ttl\_\\duration}
\TheProperty*[dns!maradns!default\_ttl\_duration]{default\_ttl\_duration \Arg{duration}}

Sets the default time to life \Arg{duration}. The duration should be set in 
ISO 8601 format\footnote{\url{http://www.ostyn.com/standards/scorm/samples/ISOTimeForSCORM.htm}}.
For the distribution
\begin{inparaitem}
\item[\TheDistribution{ubuntu}] Ubuntu 10.04,
\item[\TheDistribution{ubuntu}] Ubuntu 12.04,
\item[\TheDistribution{ubuntu}] Ubuntu 14.04
\end{inparaitem}
the value is set to \qcode{P1D}.

%% default_refresh_duration
\TheProperty[dns:maradns:default_refresh_duration]{default\_\\refresh\_\\duration}
\TheProperty*[dns!maradns!default\_refresh\_duration]{default\_refresh\_duration \Arg{duration}}

Sets the default refresh \Arg{duration}. The duration should be set in 
ISO 8601 format.
For the distribution
\begin{inparaitem}
\item[\TheDistribution{ubuntu}] Ubuntu 10.04,
\item[\TheDistribution{ubuntu}] Ubuntu 12.04,
\item[\TheDistribution{ubuntu}] Ubuntu 14.04
\end{inparaitem}
the value is set to \qcode{PT8H}.

%% default_retry_duration
\TheProperty[dns:maradns:default_retry_duration]{default\_\\retry\_\\duration}
\TheProperty*[dns!maradns!default\_retry\_duration]{default\_retry\_duration \Arg{duration}}

Sets the default retry \Arg{duration}. The duration should be set in 
ISO 8601 format.
For the distribution
\begin{inparaitem}
\item[\TheDistribution{ubuntu}] Ubuntu 10.04,
\item[\TheDistribution{ubuntu}] Ubuntu 12.04,
\item[\TheDistribution{ubuntu}] Ubuntu 14.04
\end{inparaitem}
the value is set to \qcode{PT2H}.

%% default_expire_duration
\TheProperty[dns:maradns:default_expire_duration]{default\_\\expire\_\\duration}
\TheProperty*[dns!maradns!default\_expire\_duration]{default\_expire\_duration \Arg{duration}}

Sets the default expire \Arg{duration}. The duration should be set in 
ISO 8601 format.
For the distribution
\begin{inparaitem}
\item[\TheDistribution{ubuntu}] Ubuntu 10.04,
\item[\TheDistribution{ubuntu}] Ubuntu 12.04,
\item[\TheDistribution{ubuntu}] Ubuntu 14.04
\end{inparaitem}
the value is set to \qcode{P4D}.

%% default_minimum_duration
\TheProperty[dns:maradns:default_minimum_duration]{default\_\\minimum\_\\duration}
\TheProperty*[dns!maradns!default\_minimum\_duration]{default\_minimum\_duration \Arg{duration}}

Sets the default minimum TTL \Arg{duration}. The duration should be set in 
ISO 8601 format.
For the distribution
\begin{inparaitem}
\item[\TheDistribution{ubuntu}] Ubuntu 10.04,
\item[\TheDistribution{ubuntu}] Ubuntu 12.04,
\item[\TheDistribution{ubuntu}] Ubuntu 14.04
\end{inparaitem}
the value is set to \qcode{P1D}.

%% default_mx_priority
\TheProperty[dns:maradns:default_mx_priority]{default\_mx\_\\priority}
\TheProperty*[dns!maradns!default\_mx\_priority]{default\_mx\_priority \Arg{priority}}

Sets the default MX record priority \Arg{priority}.
For the distribution
\begin{inparaitem}
\item[\TheDistribution{ubuntu}] Ubuntu 10.04,
\item[\TheDistribution{ubuntu}] Ubuntu 12.04,
\item[\TheDistribution{ubuntu}] Ubuntu 14.04
\end{inparaitem}
the value is set to \qcode{10}.
