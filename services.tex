\section{Services}

Die Services werden durch Scripte eingerichtet. In den Scripten steht eine
Beschreibung dessen, was man eingerichtet haben möchte, nicht wie. 

Als Beispiel,
ein Httpd Servicescript enthält Anweisungen verschiedene Domains einzurichten,
die verschiedene Seiten liefern. Wir beschreiben nicht wie man den Apache2
Webserver einrichtet, sondern nur die Anweisungen, dass wir diese Domains gerne
hätten. Es ist die Aufgabe des \sscontrol{} die Konfiguration für den Apach2
Webserver zu erstellen und ihn einzurichten. Diese Aufgabe löst \sscontrol{} mit
Hilfe der Profile.

Dadurch dass wir diese Konfiguration abstrahieren können wir das benutzte Profil
austauschen. Z.B. Profil A enthält Anweisungen wie man einen Ubuntu Server mit
dem Apache2 Webserver einrichtet und Profil B enthält Anweisungen wie man einen
RedHat Server mit Lighthttpd einrichtet. Man kann nun Profil A durch Profil B
ersetzen und \sscontrol{} wird die gleichen Domains mit den gleichen Seiten
einrichten.

\subsection{Hostname Service}

Der Hostname wird von verschiedenen Services verwenden um den Namen der Maschine
zu erhalten. Mit diesem Service kann man diesen Namen setzen lassen.

\subsubsection{Hostname Profil}

Das Hostname Profil wird eingeleitet durch die \directive{hostname}. Gefolgt von
den folgenden drei Direktiven:

\paragraph{\directive{configuration\_directory}}

beschreibt in welchem Verzeichnis
sich die Konfiguration für den Hostnamen befinden. Unter Ubuntu und Debian
Systemen wird die Konfiguration in der Datei \code{/etc/hostname} gespeichert,
somit ist das Verzeichnis \code{/etc}.

\paragraph{\directive{restart\_command}}

enthält das Kommando mit dem man im
laufendem Betrieb den Hostnamen ändern kann.  Unter Ubuntu und Debian
Systemen ist es das Kommando 
\code{/etc/init.d/hostname restart}.

\paragraph{\directive{status\_command}}

enthält das Kommando mit dem man den
aktuellen Hostnamen abfragen kann. Unter Ubuntu und Debian Systemen ist es das
Kommando \code{/bin/hostname}.

\begin{lstlisting}[style=Java, caption=Beispiel Hostname Profil Ubuntu Server]
profiles {
    create_profile("ubuntu") {
        hostname {
            configuration_directory '/etc'
            restart_command '/etc/init.d/hostname restart'
            status_command '/bin/hostname'
        }
    }
}
\end{lstlisting}

\subsubsection{Hostname Script}

Das Script enthält nur eine Direktive, der Name des Hosts.

\paragraph{\directive{hostname}}

erwartet eine einzige Zeichenfolge, der Name des Hosts.

\begin{lstlisting}[style=Java, caption=Beispiel Hostname Script]
server {
	hostname 'srv1'
}
\end{lstlisting}


\subsection{Hosts Service}

Unter Linux Systemen ist es üblich in der \code{/etc/hosts} Datei bekannte
Hostnamen und deren IP-Adressen zu speichern. Mit dieser Datei kann das System
Hostnamen auflösen ohne einen DNS-Server.

Mit dem Hosts Service können wir diese statische Liste der bekannten Hostnamen und
deren IP Adresse konfigurieren lassen.

\subsubsection{Hosts Profil}

Das Hostname Profil wird eingeleitet durch die \directive{hosts}. Gefolgt von
den folgenden zwei Direktiven:

\paragraph{\directive{configuration\_directory}}

beschreibt in welchem Verzeichnis
sich die Konfiguration für die Hosts befinden. Unter Ubuntu und Debian
Systemen wird die Konfiguration in der Datei \code{/etc/hosts} gespeichert,
somit ist es das Verzeichnis \code{/etc}.

\paragraph{\directive{status\_command}}

enthält das Kommando mit dem man die eingestellten Hosts abfragen kann. Es gibt
eigentlich kein Kommando, dass so einen Status abfragen kann da es nur normalerweise
nur die Datei \code{/etc/hosts} gibt. Wir können uns daher dem Kommando \code{cat}
bedienen, dass den Inhalt der Datei \code{/etc/hosts} auslesen kann.

\begin{lstlisting}[style=Java, caption=Beispiel Hosts Profil Ubuntu Server]
profiles {
	create_profile("ubuntu") {
		hostname {
			configuration_directory '/etc'
			status_command '/bin/cat /etc/hosts'
		}
	}
}
\end{lstlisting}

\subsubsection{Hosts Script}

Das Script enthält ein oder mehrere \directive{host}, dass ein oder mehrere
\directive{alias} anthalten kann.

\paragraph{\directive{hosts}}

Repräsentiert einen neuen Host mit der dazugehörigen IP Adresse. Braucht die zwei
Parameter:

\begin{asparadesc}
\item[\code{ip}] eine Zeichenfolge mit der IP Adresse des Host und
\item[\code{hostname}] der Name des Host, als eine Zeichenfolge.
\end{asparadesc}

\paragraph{\directive{alias}}

eine Zeichenfolge, die den Pseudonym des Host angibt.

\subsubsection*{Beispiel Listing}

\begin{lstlisting}[style=Java, caption=Beispiel Hosts Script]
server {
	host ip: '127.0.0.1', hostname: 'localhost'
	host(ip: '192.168.0.49', hostname: 'srv1a.globalscalingsoftware.com') {
		alias 'srv1'
		alias 'srv1a'
	}
	host(ip: '192.168.0.50', hostname: 'srv1b.globalscalingsoftware-projects.com') {
		alias 'srv1b'
	}
}
\end{lstlisting}

\begin{asparadesc}
\item[Zeile 2] zeigt die einfachste Form der \directive{host} an. Es wird ein Hosteintrag
erstellt, mit dem Hostnamen ``localhost'' und der IP Adresse ``127.0.0.1'';
\item[Zeile 3 und 6] zeigt die die \directive{host} mit einem oder mehreren Pseudonymen für einen Host.
\end{asparadesc}


\subsection{Database Service}

Mit diesem Service kann man einen Datenbankserver, wie MySQL oder PostgreSQL,
einrichten. Der Service kann einen Datenbankserver installieren, Datenbanken
und Benutzer einrichten.

\subsubsection{Database Script}

Das Hostname Profil wird eingeleitet durch die \directive{database}. Gefolgt von
den folgenden vier Direktiven:

\paragraph{\directive{bind\_address}}

die IP Adresse, an die der Server gebunden wird. Der Server empfängt nur auf diese
Adresse Anfragen. Die Adresse kann entweder eine IPv4, IPv6 Adresse oder
einen Hostnamen sein.

Die Direktive erwartet die Adresse als den ersten Parameter als eine Zeichenfolge.

\paragraph{\directive{admin\_password}}

der Administratorpasswort des Datenbankservers. Der Administrator ist der Benutzer,
der vollen Zugriff auf den Server hat. Der Database-Service wird beim installieren
des Datenbankservers dieses Passwort setzen und verwenden um Datenbanken und Benutzer
zu erstellen.

Die Direktive erwartet das Passwort als den ersten Parameter als eine Zeichenfolge.

\paragraph{\directive{database}}

bezeichnet eine neue Datenbank. Die Datenbank wird auf dem Server erstellt, falls
sie noch nicht existiert.

Die Direktive erwartet den Namen der Datenbank als den ersten Parameter als eine Zeichenfolge.

\paragraph{\directive{user}}

bezeichnet einen neuen Benutzer. Der Benutzer wird auf dem Server erstellt, falls er
noch nicht existiert. Die Direktive kann ein oder mehrere \directive{use\_database}
enthalten, die bezeichnen für welche Datenbank der Benutzer zugriff haben soll.

Die Direktive erwartet zwei Parameter:

\begin{asparadesc}
\item[\code{name}] der Name des Benutzers, als eine Zeichenfolge und
\item[\code{password}] der Password des neuen Benutzers, als eine Zeichenfolge.
\end{asparadesc}

Ein weitere optionales Parameter kann angegeben werden:

\begin{asparadesc}
\item[\code{server}] der Servername des Benutzers, als eine Zeichenfolge. Standardeinstellung
ist ``localhost''.
\end{asparadesc}

\paragraph{\directive{use\_database}}

bezeichnet die Datenbank, die auf die der Benutzer vollen Zugriff haben soll. Dem Benutzer
werden alle üblichen Rechte zugewiesen. Die Datenbank muss entweder mit der \directive{database}
erstellt werden oder muss bereits existieren.

Erwartet den Namen der Datenbank als eine Zeichenfolge im ersten Parameter.

\subsubsection*{Beispiel Listing}

\begin{lstlisting}[style=Java, caption=Beispiel Database Script]
server {
	database {
		bind_address '127.0.0.1'

		admin_password 'mysqladminpassword'

		database 'drupal6db'

		user name: 'test1', password: 'user1password'

		user name: 'test2', password: 'user1password', server: "srv1"

		user(name: 'drupal6', password: 'user1password', server: "localhost") { 
			use_database 'drupal6db' 
		}
	}
}
\end{lstlisting}

\begin{asparadesc}
\item[Zeile 3] Bindet den Datenbankserver an die Adresse ``127.0.0.1'';
\item[Zeile 5] Setzt das Administratorpasswort des Servers;
\item[Zeile 7] Erstellt eine neue Datenbank;
\item[Zeile 9] Erstellt einen neuen Benutzer an ``localhost'' (implizit);
\item[Zeile 11] Erstellt einen neuen Benutzer an ``srv1'';
\item[Zeile 13] Erstellt einen neuen Benutzer an ``localhost'' (explizit) und
erlaubt alle üblichen Rechte an der Datenbank ``drupal6db'';
\end{asparadesc}



