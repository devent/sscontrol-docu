\section{Ziele}

Das erste Ziel von Simple Server Control ist es einen Server vollständig
durch ein einziges Script einzurichten. Das Script sollte vom Menschen leicht
verständlich sein und leicht zu bearbeiten sein. Desweiteren soll man durch das
Aktuallisieren des Scriptes den Server anpassen können. Das Server Script kann
in mehrere Dateien aufgeteilt sein zur besseren Übersicht.

Dieses Server Script soll verschiedene Linux Systeme einstellen können,
vorrausgesetzt die Services sind installiert. Es soll die generischen Services
einstellen können: Hosts, Hostname, Network, Firewall, HTTP Server, FTP Server,
SMTP Server, IMAP/POP3 Server. Es soll keinen Unterschied machen ob auf einem
bestimmten Server Apache oder lighttpd installiert ist, das Script soll beide
konfigurieren können. Z.B. mit dem FTP Server: Vsftpd oder Pure-FTPd; SMTP
Server: Postfix oder Sendmail; IMAP/POP3 Server: Dovecot oder Cyrus; usw.

Im Skript muss auch eine Profil des Systems enthalten sein. Dieses
Profil enthält: Welche Services installiert sind (Apache/lighttpd,
Vsftpd/Pure-FTPd, Postfix/Sendmail, usw.), wo sich die Konfiguration befindet
(Fedora hat die Konfiguration in anderen Verzeichnisen als Ubuntu).

Wir behalten alle Script in einem Verzeichnis, die Scripte werden nach den
Services aufgeteilt. Z. B. ein Scriptdatei für Hosts, Hostname und Network, für
Firewall, für den HTTP Server, für den FTP Server, usw.

\begin{asparaitem}
\item \code{network.groovy}
\item \code{firewall.groovy}
\item \code{http.groovy}
\item \code{http-website1.groovy}
\item \code{http-website2.groovy}
\item \code{ftp.groovy}
\item \code{mail.groovy}
\end{asparaitem}

Im Falle des HTTP Servers werden wir das Script noch weiter aufteilen. Eine
Scriptdatei wird für die globale Konfiguration und eine Scriptdatei pro
Webseite.

\begin{asparaitem}
\item \code{ubuntu\_profile.groovy}
\item \code{fedora\_profile.groovy}
\end{asparaitem}
