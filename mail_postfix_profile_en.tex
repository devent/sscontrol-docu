\label{sec:postfix_profile}
\subsection{Postfix - Profile}

The postfix MTA comes in the distributions 
\begin{compactitem}
\item[\TheDistribution{ubuntu}] Ubuntu 10.04 in version 2.7.0 in the main repository;
\item[\TheDistribution{ubuntu}] Ubuntu 12.04,
\item[\TheDistribution{debian}] Debian,
\end{compactitem}

\TheProperty[mail:postfix:service]{service}
\TheProperty*[mail!postfix!service]{service \Arg{name}}

The \Arg{name} of the postfix service. Can be one of the following 
service types.
\begin{asparaitem}
\item[\qcode{postfix}:] 
the normal postfix service. Supports shared domains, UNIX system accounts;
separate domains, Unix system accounts; separate domains, non-UNIX accounts.

\item[\qcode{postfix-mysql}:] supports the same types as the normal 
postfix service but with MySQL as the back-end database.
\end{asparaitem}

\TheProperty[mail:postfix:postalias_command]{postalias\_\\command}
\TheProperty*[mail!postfix!postalias\_command]{postalias\_command \Arg{command}}

The \Arg{command} for the \code{postalias} command. Can be a full path or
just the command name that can be found in the current search path.
For the distribution
\begin{inparaitem}
\item[\TheDistribution{ubuntu}] Ubuntu 10.04
\end{inparaitem}
the value is set to \qcode{/usr/sbin/postalias}.

\TheProperty[mail:postfix:postmap_command]{postmap\_\\command}
\TheProperty*[mail!postfix!postmap\_command]{postmap\_command \Arg{command}}

The \Arg{command} for the \code{postmap} command. Can be a full path or
just the command name that can be found in the current search path.
For the distribution
\begin{inparaitem}
\item[\TheDistribution{ubuntu}] Ubuntu 10.04
\end{inparaitem}
the value is set to \qcode{/usr/sbin/postmap}.

\TheProperty[mail:postfix:mailname_file]{mailname\_\\file}
\TheProperty*[mail!postfix!mailname\_file]{mailname\_file \Arg{path}}

The \Arg{path} of the \code{mailname} file. Should be the full path 
of the file.
For the distribution
\begin{inparaitem}
\item[\TheDistribution{ubuntu}] Ubuntu 10.04
\end{inparaitem}
the value is set to \qcode{/etc/mailname}.

\TheProperty[mail:postfix:main_file]{main\_file}
\TheProperty*[mail!postfix!main\_file]{main\_file \Arg{path}}

The \Arg{path} of the main configuration file of postfix.
If the path is not absolute then the path is assumed to be
in the configuration directory.
Default is the file \qcode{main.cf} in the configuration directory.

\TheProperty[mail:postfix:master_file]{master\_file}
\TheProperty*[mail!postfix!master\_file]{master\_file \Arg{path}}

The \Arg{path} of the the master file of postfix configures
the postfix master process. 
If the path is not absolute then the path is assumed to be
in the configuration directory.
Default is the file \qcode{master.cf} in the configuration directory.

\TheProperty[mail:postfix:alias_domains_file]{alias\_domains\_\\file}
\TheProperty*[mail!postfix!alias\_domains\_file]{alias\_domains\_file \Arg{path}}

The \Arg{path} of the the alias domains hash table file. It is used to loop-up
virtual domains. If the path is not absolute then the path is assumed to be
in the configuration directory. 
Default is the file \qcode{alias\_domains} in the configuration directory.

%% alias_maps_file
\TheProperty[mail:postfix:alias_maps_file]{alias\_maps\_file}
\TheProperty*[mail!postfix!alias\_maps\_file]{alias\_maps\_file \Arg{path}}

The \Arg{path} of the the alias maps hash table file. It is used to loop-up
virtual aliases. If the path is not absolute then the path is assumed to be
in the configuration directory. 
Default is the file \qcode{alias\_maps} in the configuration directory.

%% mailbox_maps_file
\TheProperty[mail:postfix:mailbox_maps_file]{mailbox\_maps\_\\file}
\TheProperty*[mail!postfix!mailbox\_maps\_file]{mailbox\_maps\_file \Arg{path}}

The \Arg{path} of the the mailbox maps hash table file. It is used to loop-up
mailbox location for virtual users. 
If the path is not absolute then the path is assumed to be in the 
configuration directory. 
Default is the file \qcode{mailbox\_maps} in the configuration directory.

%% configuration_directory
\TheProperty[mail:postfix:configuration_directory]{configuration\_\\directory}
\TheProperty*[mail!postfix!configuration\_directory]{configuration\_directory \Arg{path}}

The \Arg{path} of the configuration directory of the postfix service. 
For the distribution
\begin{inparaitem}
\item[\TheDistribution{ubuntu}] Ubuntu 10.04
\end{inparaitem}
the value is set to \qcode{/etc/postfix}.

%% mailbox_base_directory
\TheProperty[mail:postfix:mailbox_base_directory]{mailbox\_base\_\\directory}
\TheProperty*[mail!postfix!mailbox\_base\_directory]{mailbox\_base\_directory \Arg{path}}

The \Arg{path} of the base directory for virtual mailbox.
In the directory the mailbox is created for each virtual domain and user.
Should be an absolute path.
For the distribution
\begin{inparaitem}
\item[\TheDistribution{ubuntu}] Ubuntu 10.04
\end{inparaitem}
the value is set to \qcode{/var/mail/vhosts}.

%% packages
\TheProperty[mail:postfix:packages]{packages}
\TheProperty*[mail!postfix!packages]{packages \Arg{packages}}

The \Arg{packages} list that needed to be installed for the postfix service.
For the distribution
\begin{inparaitem}
\item[\TheDistribution{ubuntu}] Ubuntu 10.04
\end{inparaitem}
the value is set to \qcode{postfix}.

%% default_destinations
\TheProperty[mail:postfix:default_destinations]{default\_\\destinations}
\TheProperty*[mail!postfix!default\_destinations]{default\_destinations \Arg{domains}}

List of \Arg{domains} that are appended to the destinations.
Caution must be taked with the domain names that contains backslashes \qcode{\textbackslash} and 
dollar signs \qcode{\$}. Dollar signs may be treated as references to 
captured subsequences and backslashes are used to escape literal characters  in the domain name.
The value is set to \qcode{\textbackslash\$myhostname, localhost.\textbackslash\$mydomain, localhost}.

%% banner
\TheProperty[mail:postfix:banner]{banner}
\TheProperty*[mail!postfix!banner]{banner \Arg{text}}

Sets the banner \Arg{text} that follows the 220 status code in the SMTP 
greeting banner. The text can contain service specific variables. Caution
must be taked with banner text that contains backslashes \qcode{\textbackslash} and 
dollar signs \qcode{\$}. Dollar signs may be treated as references to 
captured subsequences and backslashes are used to escape literal characters 
in the banner text.
For the distribution
\begin{inparaitem}
\item[\TheDistribution{ubuntu}] Ubuntu 10.04
\end{inparaitem}
the value is set to \qcode{\textbackslash\$myhostname ESMTP \textbackslash\$mail\_name}.

%% mailbox_base_pattern
\TheProperty[mail:postfix:mailbox_pattern]{mailbox\_pattern}
\TheProperty*[mail!postfix!mailbox\_pattern]{mailbox\_pattern \Arg{pattern}}

Sets the \Arg{pattern} for the directories that are created for each virtual
domain and virtual user under the mailbox base directory. 
Various variables can be used to specify the pattern.
Each of the variables needs to be eclosed in \qcode{\$\{\}}.
\begin{compactitem}
\item[\qcode{domain}:] the domain name of the virtual user;
\item[\qcode{user}:] the user name of the virtual user;
\end{compactitem}
Default is set the \qcode{\$\{domain\}/\$\{user\}} pattern.

%% minimum_uid
\TheProperty[mail:postfix:minimum_uid]{minimum\_uid}
\TheProperty*[mail!postfix!minimum\_uid]{minimum\_uid \Arg{uid}}

Sets the minimum user identification number \Arg{uid} for the virtual user.
For the distribution
\begin{inparaitem}
\item[\TheDistribution{ubuntu}] Ubuntu 10.04
\end{inparaitem}
the value is set to \code{5000}.

%% virtual_uid
\TheProperty[mail:postfix:virtual_uid]{virtual\_uid}
\TheProperty*[mail!postfix!virtual\_uid]{virtual\_uid \Arg{uid}}

Sets the user identification number \Arg{uid} for virtual users.
For the distribution
\begin{inparaitem}
\item[\TheDistribution{ubuntu}] Ubuntu 10.04
\end{inparaitem}
the value is set to \code{5000}.

%% virtual_gid
\TheProperty[mail:postfix:virtual_gid]{virtual\_gid}
\TheProperty*[mail!postfix!virtual\_gid]{virtual\_gid \Arg{gid}}

Sets the group identification number \Arg{gid} for virtual users.
For the distribution
\begin{inparaitem}
\item[\TheDistribution{ubuntu}] Ubuntu 10.04
\end{inparaitem}
the value is set to \code{5000}.

